\documentclass{sig-alternate}
%\documentclass[conference]{IEEEtran}
%\documentclass[conference,final]{IEEEtran}

\input{head}
\newif\ifdraft
\drafttrue
\ifdraft
\usepackage{xcolor}
\newcommand{\onote}[1]{ {\textcolor{cyan} { (***Ole: #1) }}}
\newcommand{\terminology}[1]{ {\textcolor{red} {(Terminology used: \textbf{#1}) }}}
\newcommand{\owave}[1]{ {\cyanuwave{#1}}}
\newcommand{\jwave}[1]{ {\reduwave{#1}}}
\newcommand{\alwave}[1]{ {\blueuwave{#1}}}
\newcommand{\jhanote}[1]{ {\textcolor{red} { ***shantenu: #1 }}}
\newcommand{\alnote}[1]{ {\textcolor{green} { ***andreL: #1 }}}
\newcommand{\amnote}[1]{ {\textcolor{blue} { ***andreM: #1 }}}
\newcommand{\smnote}[1]{ {\textcolor{brown} { ***sharath: #1 }}}
\newcommand{\pmnote}[1]{ {\textcolor{brown} { ***Pradeep: #1 }}}
\newcommand{\msnote}[1]{ {\textcolor{cyan} { ***mark: #1 }}}
\newcommand{\mrnote}[1]{ {\textcolor{purple} { ***melissa: #1 }}}
\definecolor{orange}{rgb}{1,.5,0}
\newcommand{\aznote}[1]{ {\textcolor{orange} { ***ashley: #1 }}}
\definecolor{dandelion}{cmyk}{0,0.29,0.84,0}
\newcommand{\mtnote}[1]{ {\textcolor{dandelion} { ***matteo: #1 }}}
\newcommand{\note}[1]{ {\textcolor{magenta} { ***Note: #1 }}}
\else
\newcommand{\onote}[1]{}
\newcommand{\terminology}[1]{}
\newcommand{\owave}[1]{#1}
\newcommand{\jwave}[1]{#1}
\newcommand{\alnote}[1]{}
\newcommand{\amnote}[1]{}
\newcommand{\aznote}[1]{}
\newcommand{\athotanote}[1]{}
\newcommand{\smnote}[1]{}
\newcommand{\pmnote}[1]{}
\newcommand{\jhanote}[1]{}
\newcommand{\msnote}[1]{}
\newcommand{\mtnote}[1]{}
\newcommand{\note}[1]{}
\fi

\newcommand{\cloud}{cloud\xspace}
\newcommand{\clouds}{clouds\xspace}
\newcommand{\pilot}{Pilot\xspace}
\newcommand{\pilots}{Pilots\xspace}
\newcommand{\pilotjob}{Pilot-Job\xspace}
\newcommand{\pilotjobs}{Pilot-Jobs\xspace}
\newcommand{\pilotcompute}{Pilot-Compute\xspace}
\newcommand{\pilotcomputes}{Pilot-Computes\xspace}
\newcommand{\pilotdata}{Pilot-Data\xspace}
\newcommand{\pilotdataservice}{Pilot-Data Service\xspace}
\newcommand{\pilotcomputeservice}{Pilot-Compute Service\xspace}
\newcommand{\computedataservice}{Compute-Data Service\xspace}
\newcommand{\pilotmapreduce}{PilotMapReduce\xspace}
\newcommand{\mrmg}{MR-Manager\xspace}
\newcommand{\pstar}{P*\xspace}
\newcommand{\pd}{PD\xspace}
\newcommand{\pj}{PJ\xspace}
\newcommand{\pjs}{PJs\xspace}
\newcommand{\pds}{Pilot Data Service\xspace}
\newcommand{\computeunit}{Compute-Unit\xspace}
\newcommand{\computeunits}{Compute-Units\xspace}
\newcommand{\dataunit}{Data-Unit\xspace}
\newcommand{\dataunits}{Data-Units\xspace}
\newcommand{\du}{DU\xspace}
\newcommand{\dus}{DUs\xspace}
\newcommand{\cu}{CU\xspace}
\newcommand{\cus}{CUs\xspace}
\newcommand{\su}{SU\xspace}
\newcommand{\sus}{SUs\xspace}
\newcommand{\schedulableunit}{Schedulable Unit\xspace}
\newcommand{\schedulableunits}{Schedulable Units\xspace}
\newcommand{\cc}{c\&c\xspace}
\newcommand{\CC}{C\&C\xspace}
\newcommand{\up}{\vspace*{-1em}}
\newcommand{\upp}{\vspace*{-0.5em}}
\newcommand{\numrep}{8 }
\newcommand{\samplenum}{4 }
\newcommand{\tmax}{$T_{max}$ }
\newcommand{\tc}{$T_{C}$ }
\newcommand{\tcnsp}{$T_{C}$}
\newcommand{\bj}{BigJob\xspace}
\newcommand{\MW}{Master-Worker\xspace}

\newcommand{\I}[1]{\textit{#1}\xspace}
\newcommand{\B}[1]{\textbf{#1}\xspace}
\newcommand{\T}[1]{\texttt{#1}\xspace}
\newcommand{\C}[1]{\textsc{#1}\xspace}

\lstdefinestyle{myListing}{
  frame=single,   
  backgroundcolor=\color{listinggray},  
  %float=t,
  language=C,       
  basicstyle=\ttfamily \footnotesize,
  breakautoindent=true,
  breaklines=true
  tabsize=2,
  captionpos=b,  
  aboveskip=0em,
  belowskip=-2em,
  %numbers=left, 
  %numberstyle=\tiny
}      

\lstdefinestyle{myPythonListing}{
  frame=single,   
  backgroundcolor=\color{listinggray},  
  %float=t,
  language=Python,       
  basicstyle=\ttfamily \footnotesize,
  breakautoindent=true,
  breaklines=true
  tabsize=2,
  captionpos=b,  
  %numbers=left, 
  %numberstyle=\tiny
}



%  \setlength{\parskip}{0.05ex} % 1ex plus 0.5ex minus 0.2ex}
%  \setlength{\parsep}{0pt}
%  %\setlength{\headsep}{0pt}
%  \setlength{\topskip}{0pt}
%  \setlength{\topmargin}{0pt}
%  %\setlength{\topsep}{0pt}
%  \setlength{\partopsep}{0pt}

% This is now the recommended way for checking for PDFLaTeX:


\ifpdf
\DeclareGraphicsExtensions{.pdf, .jpg, .tif}
\else
\DeclareGraphicsExtensions{.eps, .jpg, .ps}
\fi

\tolerance=1000
\hyphenpenalty=10

\usepackage{lscape}

\usepackage{listings}

\lstnewenvironment{code}[1][]%
{
\noindent
%\minipage{0.98 \linewidth}
\minipage{1.0 \linewidth}
\vspace{0.5\baselineskip}
\lstset{
    language=Python,
%    numbers=left,
%    numbersep=4pt,
    frame=single,
    captionpos=b,
    stringstyle=\ttfamily,
    basicstyle=\scriptsize\ttfamily,
    showstringspaces=false,#1}
}
{\endminipage}

\begin{document}
\conferenceinfo{HPDC'13}{2013, New York, USA}
% \conferenceinfo{ECMLS'11,} {June 8, 2011, San Jose, California, USA.}
% \CopyrightYear{2011}
% \crdata{978-1-4503-0702-4/11/06}
% \clubpenalty=10000
% \widowpenalty = 10000

\title{A Fresh Perspective on Pilot-Jobs}

% \alignauthor
% Ben Trovato\titlenote{Dr.~Trovato insisted his name be first.}\\
%        \affaddr{Institute for Clarity in Documentation}\\
%        \affaddr{1932 Wallamaloo Lane}\\
%        \affaddr{Wallamaloo, New Zealand}\\
%        \email{trovato@corporation.com}
% % 2nd. author
% \alignauthor
% G.K.M. Tobin\titlenote{The secretary disavows
% any knowledge of this author's actions.}\\
%        \affaddr{Institute for Clarity in Documentation}\\
%        \affaddr{P.O. Box 1212}\\
%        \affaddr{Dublin, Ohio 43017-6221}\\
%        \email{webmaster@marysville-ohio.com}
% % 3rd. author
% \alignauthor Lars Th{\o}rv{\"a}ld\titlenote{This author is the
% one who did all the really hard work.}\\
%        \affaddr{The Th{\o}rv{\"a}ld Group}\\
%        \affaddr{1 Th{\o}rv{\"a}ld Circle}\\
%        \affaddr{Hekla, Iceland}\\
%        \email{larst@affiliation.org}
% \and  % use '\and' if you need 'another row' of author names
% % 4th. author
% \alignauthor Lawrence P. Leipuner\\
%        \affaddr{Brookhaven Laboratories}\\
%        \affaddr{Brookhaven National Lab}\\
%        \affaddr{P.O. Box 5000}\\
%        \email{lleipuner@researchlabs.org}
% % 5th. author
% \alignauthor Sean Fogarty\\
%        \affaddr{NASA Ames Research Center}\\
%        \affaddr{Moffett Field}\\
%        \affaddr{California 94035}\\
%        \email{fogartys@amesres.org}
% % 6th. author
% \alignauthor Charles Palmer\\
%        \affaddr{Palmer Research Laboratories}\\
%        \affaddr{8600 Datapoint Drive}\\
%        \affaddr{San Antonio, Texas 78229}\\
%        \email{cpalmer@prl.com}
% }

\date{}
\maketitle

\begin{abstract}
  There is no agreed upon definition of \pilotjobs; however a
  functional attribute of \pilotjobs that is generally agreed upon is
  they are tools/services that support multi-level and/or
  application-level scheduling by providing a scheduling overlay on
  top of the system-provided schedulers.  
  Nearly everything else is either specific to an
  implementation, open to interpretation or not agreed upon. For
  example, are \pilotjobs part of the application space, or part of
  the services provided by an infrastructure? We will see that
  close-formed answers to questions such as whether \pilotjobs are
  system-level or application-level capabilities are likely to be
  elusive. Hence, this paper does not make an attempt to provide
  close-formed answers, but aims to provide appropriate context,
  insight and analysis of a large number of \pilotjobs, and thereby
  bring about a hitherto missing consilience in the community's
  appreciation of \pilotjobs.  Specifically this paper aims to provide
  a comprehensive survey of \pilotjobs, or more generically of
  \pilotjob like capabilities.  A primary motivation for this work stems
  from our experience when looking for an interoperable, extensible
  and general-purpose \pilotjob; in the process, we realized that
  such a capability did not exist. The situation was however even more
  unsatisfactory: in fact there was no agreed upon definition or
  conceptual framework of \pilotjobs.  To substantiate these points of
  view, we begin by sampling (as opposed to a comprehensive survey)
  ~\onote{a few lines above we say that we're doing a comprehensive
    survey!} some existing \pilotjobs and the different aspects of
  these \pilotjobs, such as the applications scenarios that they have
  been used and how they have been used. The limited but sufficient
  sampling highlights the variation, and also provides both a
  motivation and the basis for developing an implementation agnostic
  terminology and vocabulary to understand \pilotjobs; Section \S3
  attempts to survey the landscape/eco-system of \pilotjobs.  With an
  agreed common framework/vocabulary to discuss and describe
  \pilotjobs, we proceed to analyze the most commonly utilized
  \pilotjobs and in the process provide a comprehensive survey of
  \pilotjobs, insight into their implementations, the infrastructure
  that they work on, the applications and application execution modes
  they support, and a frank assessment of their strengths and
  limitations.  An inconvenient but important question -- both
  technically and from a sustainability perspective that must be
  asked: why are there so many similar seeming, but partial and
  slightly differing implementations of \pilotjobs, yet with very
  limited interoperability amongst them?  Examining the reasons for
  this state-of-affairs provides a simple yet illustrative case-study
  to understand the state of the art and science of tools, services
  and middleware development.  Beyond the motivation to understand the
  current landscape of \pilotjobs from both a technical and a
  historical perspective, we believe a survey of \pilotjobs is a
  useful and timely undertaking as it provides interesting insight
  into understanding issues of software sustainability.
  % believe that a survey of \pilotjobs provides and appreciation for
  % the richness of the \pilotjobs landscape.  is
  % not to discuss the \pstar conceptual framework, but That led to
  % the \pstar model.
\end{abstract}

\section{Introduction}\label{sec:intro}

% \jhanote{Generally not good style to begin new subsection immediately
%   after section starting}
% \jhanote{Now develop the following paragraph along the lines of: Why
%   have \pilotjobs been successful?}

% \jhanote{Although pilotjobs have solved/addressed many problems, now
%     develop the problem with \pilotjobs themselves..}

The seamless uptake of distributed infrastructures by scientific
applications has been limited by the lack of pervasive and
simple-to-use abstractions at multiple levels - at the development,
deployment and execution stages. Of all the abstractions proposed to
support effective distributed resource utilization, a survey of actual
usage suggested that \pilotjobs were arguably one of the most
widely-used distributed computing abstractions - as measured by the
number and types of applications that use them, as well as the number
of production distributed cyberinfrastructures that support them.
\msnote{ref?}

The fundamental reason for the success of the \pilotjob abstraction is
that \pilotjobs liberate applications/users from the challenging
requirement of mapping specific tasks onto explicit heterogeneous and
dynamic resource pools.  In other words, at least in part, due to the
decoupling between task/workload specification and task
management. \pilotjobs also improve the efficiency of task assignment
and shield applications from having to load-balance tasks across such
resources.\onote{not sure if 'load-balance' is appropriate here}
Another concern often addressed by \pilotjobs is fault tolerance which
commonly refers the ability of the \pilotjob system to verify the
execution environment before executing jobs. The \pilotjob abstraction
is also a promising route to address specific requirements of
distributed scientific applications, such as coupled-execution and
application-level
scheduling~\cite{ko-efficient,DBLP:conf/hpdc/KimHMAJ10}.

%   \onote{I think the most important reasons why Pilot Jobs being so
%     popular (and re-invented over and over again) is that they allow
%     the execution of small (i.e., singe / few-core) tasks efficiently
%     on HPC infrastrucutre by massively reducing queueing time. HPC
%     sites (from schedulers to policies) have always been (and still
%     are) discrimatory against this type of workload in favor of the
%     large, tightly-coupled ones. Pilot-Jobs try to counteract. While
%     this is certainly not the main story that we want to tell, this
%     should IMHO still be mentioned. } \jhanote{This is definitely one
%     of the main reasons, but as Melissa pointed out it during RADICAL
%     call, it is by no means the only reason. Need to get the different
%     reasons down here.. then find a nice balance and description}


A variety of PJ frameworks have emerged: Condor-G/
Glide-in~\cite{condor-g}, Swift~\cite{Wilde2011},
DIANE~\cite{Moscicki:908910}, DIRAC~\cite{1742-6596-219-6-062049},
PanDA~\cite{1742-6596-219-6-062041}, ToPoS~\cite{topos},
Nimrod/G~\cite{10.1109/HPC.2000.846563}, Falkon~\cite{1362680} and
MyCluster~\cite{1652061} to name a few. Although they are all, for the
most part, functionally equivalent -- they support the decoupling of
workload submission from resource assignment -- it is often impossible
to use them interoperably or even just to compare them functionally or
qualitatively.  The situation is reminiscent of the proliferation of
functionally similar yet incompatible workflow systems, where in spite
of significant a posteriori effort on workflow system extensibility
and interoperability (thus providing post-facto justification of its
needs), these objectives remain difficult if not infeasible.

The remainder of this paper is as follows yadayadayada.
In section \ref{sec:history} we go back in time and look at how the concept of
\pilotjobs has evolved by dissecting existing \pilotjob systems and systems
with pilot-like characteristics.

\section{Functional Evolution of Pilot-Jobs}\label{sec:history}

%\subsection{A Functional Approach to Pilot-Jobs}
%Many scientific communities began running into the same issues:

% As distributed systems grew in capacity and capability, they also grew
% in complexity and heterogeneity. For example, many machines
% implemented their own batch queuing systems, and oftentimes these
% systems varied from machine to machine.\msnote{The part after the comma is kind
% of implicit by the part before}
% The wide use of heterogenous
% resources, resulted in the need for workload management across these
% resources.  In order to harness the power of these heterogeneous
% resources to run jobs, one particular solution proposed is that of
% \pilotjobs

% , which have historically been used as a means of solving these
% issues.
% This gave rise to the the need for job submission management via batch
% queuing systems and middleware access also grew.
%We briefly discuss some specific uses of \pilotjobs below.

% \pilotjobs are most commonly used for the execution of many tasks
% through the use of a container job. They are often measured by
% their throughput, that is, the number of tasks that they can complete
% per second (tps)\msnote{I dont think we use tps further in the paper},
% or alternatively, by the total number of tasks
% executed. As such, \pilotjobs are used to achieve
% high-throughput, for example, when using genome sequencing techniques
% \msnote{Arguably genome applications are not the most illustrative example of
% high throughput tasks getting benefit out of pilot-jobs}
% or ensemble-based applications. \pilotjobs have also been used for
% parameter sweeps, chained tasks, and loosely-coupled but distinct
% tasks. %note to self: cite these with papers

% Multi-scale simulations have also benefited from the use of
% \pilotjobs. A framework for load balancing via dynamic resource
% allocation for coupled multi-physics (MPI-based) simulations using
% \pilotjobs was demonstrated in Ref.~\cite{ko-efficient}.
% This was achieved by dynamically assigning more processors to
% jobs with longer runtimes, so that these jobs could accomplish their workload
% in the same amount of wall-clock time as those with shorter runtimes.
% This led to an overall reduction of jobs that were waiting to communicate
% via MPI, and an overall reduction of the total simulation runtime.

% \pilotjobs can be used for  simulations
%  with varying numbers of tasks to complete, for example,
% molecular dynamics simulations requiring task restart. These types of
%  simulations may start with a fixed number of tasks but spawn
%  more tasks in order to continue simulating. \pilotjobs can be
% utilized for these types of dynamic simulations, because
% new tasks can be fed to the \pilot at any time within a given
% runtime. Without \pilotjobs, these simulations would have to
% be resubmitted to the batch queue and wait for their time
% to become active again~\cite{luckow2009adaptive}.

% \pilotjobs have also been used to avoid queue wait times for many jobs 
% \msnote{I think we just said this above in the high-throughput case}
% as well as harness and utilize different resources (with different
% batch queueing systems) to do \textit{scale-across} simulations.
% As a fault tolerant mechanism, many \pilotjob systems monitor
% failed jobs and have the ability to restart them within the given
% time frame of the \pilotjob's total runtime~\cite{1742-6596-219-6-062049,condor-g,nilsson2011atlas}.


% In order to appreciate \pilotjobs, we outline the evolution of
% \pilot-like capabilities ultimately leading to the creation of the
% first actual \pilotjob. We present a brief chronological order of
% \pilotjob-like systems, beginning with simple Master-Worker-based
% applications through advanced workload management systems.

% \subsubsection*{The Evolution of \pilotjobs}\label{sssec:evolution}

%\pilotjobs provide the ability to distribute workload across multiple systems and
%offer an easy way to schedule many jobs at one time. This in turn improves the
%utilization of resources\msnote{why?}, reduces the net wait time of a collection of tasks, and
%also prevents saturation of resource batch queuing systems from high-throughput
%simulations where many jobs need to be run at one time\msnote{I dont understand
%the last claim}. While early \pilot-systems
%solely provided this placeholder job mechanism, many of these system evolved to
%more complex workload management systems. As applications began to utilize
%distributed cyberinfrastructure, the workloads grew from small sets of short
%running jobs to many jobs with either short or potentially long runtimes. There was
%a need for more complex management of these workloads and additional capabilities
%for user-level control of the tasks that would be executed within the placeholder
%job. This drove the creation of the modern idea of \pilots \msnote{Kind of an
%ambigous statement, like all statements that include the word "modern" :-)}

%\pilotjob systems differ in their focus and architecture. Our preliminary
%survey of existing \pilotjobs helped to identify three major layers that
%these systems exhibit: (i) core \pilotjob functionality - this provides the 
%minimally complete set of capabilities for a simple \pilotjob, (ii) advanced
%\pilotjob functionality - a system that offers all of (i) plus a more sophisticated
%resource management mechanism, and (iii) higher-level \pilot-based frameworks
%- frameworks utilize \pilots for a specific use case, e.\,g.\ workflows or data analytics. 

%\onote{These are not necessarily 'layers'. The more I think about it, the whole
%idea of 'layers' doesn't make so much sense if we want to distinguish between core and 
%advanced systems / frameworks. I think discussing these 'functionality' along 
%\textbf{orthogonal components} (that doesn't necessarily build upon each other)
%would make more sense. My (and ALs) comment w.r.t Figure 1 is related to this.}
%\msnote{I'm tempted to at this stage in the paper use a very high level figure
%to simply support the explanation of the pilotjob concept.}

%As one can intuit from the above descriptions, existing \pilotjob systems maybe
%overlap and overflow into these different layers, and each layer builds upon the
%previous one. Therefore, we use these classification layers only as a means
%to explain the basic progression of a \pilotjob system from simple scheduling
%reservation mechanisms to more complete job management systems. A 
%more semantically-rich terminology and classification scheme will be presented
%in Sections \ref{sec:vocab} and \ref{sec:4}.\msnote{I think the whole layering
%discussion can go.}


%Figure~\ref{fig:figures_classification} categorized \pilotjob systems into three
%layers: (i) core \pilotjob systems that solely provide a simple \pilot capability,
%(ii) advanced \pilotjob systems that offer sophisticated resource management
%capabilities based on \pilots, and (iii) higher level \pilot-based frameworks
%that utilize \pilots for a specific use case, e.\,g.\ workflows or data analytics.
%One of the important aspects of these layers is that they often overlap or
%have evolved from one another. Therefore, it is hard to classify 


% \begin{figure}[t]
%	\centering
%		\includegraphics[width=0.45\textwidth]{figures/classification}
%	\caption{Pilot-Job Classification: Different PJ systems focus
%         on different parts of the distributed computing stack: (i)
%          PJ systems that solely provide the \pilot capability, (ii)
%          systems that offer resource management capabilities based on
%          \pilots and (iii) applications, tools and services that
%          utilize \pilots for resource management. \jhanote{we should
%            make the three levels of the diagram consistent with the
%            three categories, ``core PJ'' , ``advanced PJ'' and
%            ``higher-level PJs'' . Also earlier comment about adding
%            ``higher-level pilot-based frameworks'' to ``higher-level
%            frameworks that can use pilot-jobs''.}}  \alnote{mention
%          that these layers are not cleanly separated, add
%          capabilities (outside)/properties (internal) of
%          each layer, what is the overlap between the layers (how they
%          interrelated?, what is the overlap?)?}
%          \onote{IMHO this figure doesn't really help to explain things
%          and is also wrong (see AL's comment above). I think we should 
%          get rid of it.  } \msnote{+1}
%	\label{fig:figures_classification}
%\end{figure}

%In this section, we give a brief overview of different \pilotjob systems. As
%grid computing advanced in its size and capabilities, the need for
%job scheduling and time-sharing became more prevalent. Batch
%queuing systems were installed to solve this problem, wherein a login
%node accepted all job submissions and then the queuing system
%divided the work to the worker nodes. 
%The requirements of distributed applications, such as efficient load
%balancing and resource utilization across multiple resources, drove
%the need for user-level control of tasks and ease of use of job
%descriptions for data driven
%applications~\cite{ko-efficient}~\cite{DBLP:conf/hpdc/KimHMAJ10}, but
%the concept of a \pilot was not the first type of application-level scheduling 
%introduced.






When Lewis Fry Richardson in 1922 devised his \textit{Forecast Factory} (Figure 
\ref{fig:figures_forecast-factory}), it might have been the first
mentioning of large scale parallel computing.
In determining the necessary processing power for weather forecasting, he
estimated that 64000 \textit{computers} (human beings in this case) would be
required for solving the equations.
These \textit{computers} would all be assigned a part of the globe by a central
\textit{senior clerk}.
The \textit{computers} would perform their calculations and the results would
be collected by the clerk.
This was in effect a Master-Worker 
pattern~\cite{Shao:2000:masterslave}.\alnote{Is this really the defining moment 
of the emergence of the master/worker pattern? I would argue that this pattern 
exists since beginning of humanity....}\mrnote{I agree with AL. This is a stretch,
and at this point, I feel like I have lost the Pilot mission of this paper.}
\aznote{Agreed -- this seems like something that would fit in with a layman's
introduction in a magazine, not directly relevant.}

\begin{figure}[t]
	\centering
		\includegraphics[width=.45\textwidth]{figures/forecast-factory.jpg}
	\caption{\textit{Forecast Factory} as envisioned by Lewis Fry Richardson.
    Drawing by Fran{\c c}ois Schuiten.}
	\label{fig:figures_forecast-factory}
\end{figure}

As established in the introduction, \pilotjobs have proven to be an
effective tool for task parallelism.  (Note that the name
\textit{\pilotjob} was not used before X, and was introduced by Y.)
\mrnote{Is the above parenthetical phrase supposed to be in the text
as is? X vs. Y?}
One common use for the \pilotjob paradigm is the aforementioned
Master-Worker (M-W) scheme and its associated
frameworks. \jhanote{Mark: I propose the following: One common use for
  the M-W scheme is to serve as the coordination substrate for PJs. I
  know its a bit nebulous, but it connects the two concepts directly,
  which is the goal here}

In the context of distributed systems, the M-W scheme was initially used for
farming tasks from a master to a various number of workers, and could easily be
adapted to run in a platform-independent way across the potentially
heterogeneous resources~\cite{masterworker, Goux00anenabling}.
M-W based frameworks could respond to the dynamically changing resources by
adapting the number of workers to match the resource availability.

As the resources in distributed computing infrastructures adopted more and more
batch queuing systems, users were forced to submit their jobs individually to a
scheduler. \mrnote{I find the above sentence to be grammatically awkward, also
it does not flow right from the previous sentence, and it provides no context
as to why distributed computing infrastructures started doing this. It also
talks about these concepts as if everyone knows what they are. A proposal: 
As distributed computing infrastructures became more popular and available, 
user demand drove the need for efficient shared allocation of heterogenous
resources. Leveraging the batch processing concept, first used in the time
of punchcards, job schedulers were created to accommodate these needs, often
called ``batch queuing systems.'' The adoption of batch queuing systems in 
practice onto clusters and grids meant users were forced to submit their tasks
individually to the job scheduler (or, queue). [I would say rather than..., but I'm not 
sure the right way to complete that]} 
Often, the type of scheduler on a one machine was different than that of
another machine. There was a need for managing the heterogenous, dynamic grid environments,
especially in terms of dynamic scheduling.
\aznote{Yes -- at the very least, the term scheduling is used but not defined.}

This drove the creation of AppLeS~\cite{Berman:1996:apples}, a framework for
application-level scheduling.
For this purpose, AppLeS provides an agent that can be embedded into an
application enabling the application to acquire resources (e.\,g.\ via Globus,
SSH or Legion) and efficiently schedule tasks onto these.
Besides M-W, AppLeS provides also different application templates, e.\,g.\ for
parameter sweep and moldable parallel applications
~\cite{Berman:2003:ACG:766629.766632}.


%
% \msnote{\cite{Gehring:1996:mars} looks just as relevant and predates apples: "...
% Note: AppLeS more active than MARS now
% MARS uses accumulated statistical data on previous execution runs of the same
% application to derive an improved task-to-process mapping"}.


%
% Commented out GHS for now.
% 
% The rise of application-level scheduling, as in AppLeS, opened new
% possibilities to Grid environments. The concept of application-level
% scheduling was extended to include long-term performance prediction in
% heterogenous Grid environments via the Grid Harvest Service (GHS)
% system~\cite{ghs}. GHS provides a prediction model that was derived by
% probability analysis and simulation and useful for large-scale
% applications in shared environments. Its prediction models and task
% scheduling algorithms are utilized in the placement of tasks across
% Grid resources. GHS supports three classes of task scheduling: (i)
% single task, (ii) parallel processing, and (iii) meta-task. The
% performance evaluation and modeling in conjunction with task-specific
% management (such as placement, scheduling, and execution) allows the
% utilization of many heterogenous resources in an efficient manner.

\begin{figure}[t]
    \centering
        \includegraphics[width=0.45\textwidth]{figures/timeline}
                \caption{Introduction of terms and systems over
                  time. \jhanote{this diagram can be extended to have
                    an evolution of functional components on the lower
                    x-y plane}}
\end{figure}

\begin{figure}[t]
	\centering
		\includegraphics[width=.45\textwidth]{figures/pilotjob-clustering.pdf}
	\caption{Pilot-Job Clustering}
	\label{fig:figures_pilotjob-clustering}
\end{figure}


Although systems such as AppLeS gave user-level control of scheduling, initial
queue reservation time still was an issue.
% \msnote{Why was queue reservation time still an issue if there was user-level
% control?}
\mrnote{Is it really initial queue reservation time or is it really the aggregate 
wait time of submitting
many tasks to the same scheduler?}
\aznote{Agreed -- could maybe add something here about variable workloads/delays
in-between variable workload processes to motivate further, but that might
just overly complicate things at this point in time...}
This brought about the idea of placeholder
scheduling~\cite{Pinchak02practicalheterogeneous}.  A placeholder was
an early \pilot mechanism in that it was an abstraction layer above
the various batch queuing systems available on different resources.
It held a \textit{place} in the regular batch queue, and when it
became active, it could pull tasks to execute.  Placeholder scheduling
was advantageous in that it did not require any special super-user
privileges on the machines, which was something most grid users did
not have access to.  It also provided a means of load balancing across
the different resources.  As placeholder scheduling evolved, it came
to include dynamic monitoring and throttling of the different
placeholders based on the queue times on the machines.

%Core \pilotjob systems focus on the basic \pilot capabilities, i.\,e.\ the
%provisioning of the placeholder job capability. 
% Various Master-Worker systems
% that provide such a mechanism (e.\,g.\ Nimrod-G~\cite{10.1109/HPC.2000.846563}). 
% Condor-G/Glide-In is the most well-known \pilotjob system.  \msnote{Im
% tempted to not make claims like this, especially not without backing up.}

% Further examples for lightweight \pilotjob systems are: ToPoS~\cite{topos},
% MyCluster~\cite{Walker:2007:PAC:1285840.1285848}, MySGE~\cite{mysge},
% GridBot~\cite{Silberstein:2009:GEB:1654059.1654071} and LGI~\cite{lgi}.


%Nimrod-G~\cite{10.1109/HPC.2000.846563}, DIANE~\cite{diane-thesis} and Work
%Queue~\cite{workqueue-pyhpc2011} are examples of Master-Worker systems that
%utilize a placeholder agent that dispatches and manages tasks. For example,
%Nimrod-G utilizes a Job Wrapper that is responsible for pulling a task and its
%associated data and then manages the execution of this task. While modern
%\pilotjobs often acquire resources opportunistically and then distribute tasks
%to resources they were able to acquire, Nimrod-G utilizes a central, cost-based
%scheduler.\msnote{Is this really a distinction, arent most of the systems
%centrally controlled?}


Around the same time as AppLeS was introduced, volunteer computing
projects started using the M-W approach to achieve high-throughput calculations
for a wide range of problems.
%This also used a pull based model, as there was no central control of these
%resources.
These volunteer resources could be distributed around the world, users
merely have to download a client program, and it runs in the
background on their computer.  The volunteer workers were essentially
heterogeneous and the pool of workers more dynamic, as opposed to
essentially homogeneous and static in AppLeS. The idea of farming out
tasks in a distributed environment including personal computers was
powerful in that it essentially made a computing grid out of many less
powerful machines.

The first public volunteer computing project was The Great Internet Mersenne
Prime Search effort\cite{woltman:2004:gimps}, shortly followed by
distributed.net~\cite{Lawton:2000:distributednet} in 1997 to attack the RC5-56
secret-key challenge.
The third large volunteer computing project was the SETI@Home project, which
set out to analyze radio telescope data.
Out of this last project, the generic BOINC distributed master-worker framework
grew \cite{Anderson:2004:BSP:1032646.1033223}.
Applications can be built on top of BOINC for their own scientific endeavors.
\mrnote{As just an end reader, it is really unclear how this relates to anything.
Is this supposed to follow from the previous para?}
\aznote{Yes -- perhaps combine this + last para?  maybe even next para too...}

While in the AppLes and placeholder job scenarios the resources were planned
from a central place, the resources in the volunteer computing were more of an
ad-hoc nature.
The central master would in the latter case not know which resources could come
available at any given time, and therefore there was no advance allocation of
work to a specific worker.
This in contrast with when there is an orchestrated set of resources requested
by the master, in that case the master can for example equally divide the tasks
over the allocated masters.

Condor is a high-throughput distributed batch computing system. Originally
Condor was made for systems within one administrative domain.
With Flocking~\cite{Epema:1996:flocking}, it was possible to group multiple
Condor systems (pools) together so that their resources could be used in an
aggregative manner.
However, this mode of operation was limited to system space by the owners of
the individual Condor systems and could be done on application level.
\mrnote{There's a better way to word this previous sentence. system-space vs 
user-space is more of a technical slang in this context. We should just
improve the readability. Also, the however doesn't seem to fit with the conjunction -
like however, it runs only in system space and this could be done on application
level? i was wondering if this should be couldn't? but not sure. anyway, maybe
just say that this mode of operation was limited in that Condor was only
available on the systems in which administrators installed the software.}
Condor-G/Glide-in~\cite{condor-g} is one of the pioneers of the \pilotjob
concept.
Glide-in is a mechanism by which a user can add remote grid resources to the
local condor pool and run their jobs on the added resource the same way that
all condor jobs are submitted.
% A Glide-in is submitted using the Condor-G grid universe \msnote{Is this a
% distinctive feature? Arent the universes just labels anyway?}.
On the remote resource a set of Condor daemons is started, which then
registers the available job slots with the central Condor pool.  The
resources added are available only for the user who added the resource
to the pool, thus giving complete control over the resources for
managing jobs without any queue waiting time.  Glide-in installs and
executes necessary Condor daemons and configuration on the remote
resource, such that the resource reports to and joins the local Condor
pool. Various systems that built on the \pilot
capabilities of Condor-G/Glide-in have been developed, e.\,g.\
Bosco~\cite{bosco}.

%Glide-in is limited in that the daemons must be running on a given resource,
%meaning that this process must be approved by resource owners or system
%administrators.

% Venus-C~\cite{venusc-generic-worker} provides a \pilotjob-like
% capability on Microsoft Azure clouds called a Generic Worker. The
% Generic Worker creates a layer of abstraction above the inner workings
% of the cloud.  The idea behind the Generic Worker is to allow
% scientists to do their science without requiring knowledge of backend
% HPC systems by offering e-Science as a service. Venus-C has not been
% shown to work with grids, because its main objective is to motivate
% scientists to use cloud infrastructures.  While the notion of moving
% to the cloud for data-driven science is an important one, many
% existing cyberinfrastructures still have powerful grid computers that
% can also be leveraged to assist with the data-driven computations.

\mrnote{What are core PJ systems? Core pilot job systems =
those that provide only simple pilot capabilities? What are
simple pilot capabilities? Not sure this has been established.}
While core PJ systems mainly focus on providing simple \pilot
capabilities commonly in application space, many of these systems
evolved towards or were extended to \pilot-based workload managers.
For example, various systems that built on the \pilot capabilities
have been developed, e.\,g.\ GlideinWMS and
GlideCAF~\cite{Belforte:2006:glidecaf}, and
PanDa~\cite{1742-6596-331-7-072069}.  These higher-level systems which
are often centrally hosted, move critical functionality from the
client to the server (i.e. a service model).  These systems usually
deploy \pilot factories that automatically start new \pilots on demand
and integrate security mechanisms to support multiple users
simultaneously.

Several of these have been developed in the context of the LHC
experiment at CERN, which is associated with a major increase in the
uptake and availability of \pilots, e.\,g.\ GlideInWMS,
DIRAC~\cite{1742-6596-219-6-062049},
PanDa~\cite{1742-6596-331-7-072069},
AliEn~\cite{1742-6596-119-6-062012} and Co-Pilot~\cite{copilot-tr}.
Each of these \pilots serves a particular user community and
experiment. Interestingly, we observe that these \pilots are
functionally very similar, work on almost the same underlying
infrastructure, and serve applications with very similar (if not
identical) characteristics.


GlideinWMS~\cite{1742-6596-119-6-062044} is a higher-level workload management
system that is based on the \pilot capabilities of Condor-G/Glide-in.
The system can, based on the current and expected number of jobs in the pool,
automatically increase or decrease the number of active Glide-ins (\pilots)
available to the pool.
GlideinWMS is a multi-user \pilotjob system commonly deployed as a hosted
service. In contrast to low-level \pilotjob systems, GlideinWMS attempts to
hide rather than expose the \pilot capabilities from the user.
GlideinWMS is currently deployed in production on the Open Science Grid
(OSG)~\cite{url_osg} and is the recommended mode for accessing OSG resources.

In comparison with the generic functionality of GlideinWMS,
PanDA~\cite{1742-6596-331-7-072069} workload management system is much
more specific to its application/community. However, it has been
extended to use Condor-G/Glidein, in addition to its native \pilot
system.  Furthermore, in contrast with a Condor-G only approach, PanDA
can utilize multiple queues; each \pilot is assigned to a certain
PanDA internal queue. PanDA also provides the ability to manage data
associated the jobs managed by the PanDA workload manager.
\aznote{I believe this paragraph should be headed differently to
focus on inter-pilot operation which is the biggest contribution
contrasted to previous paragraphs.  ex ``Multi-\pilotjob operation
is provided by systems such as PanDA, which is capable of 
using Condor-G/Glidein in addition to its native \pilot system.''}

In addition to processing, AliEn~\cite{1742-6596-119-6-062012} also provides
the ability to tightly integrate storage and compute resources and is also able
to manage file replicas.
While all data can be accessed from anywhere, the scheduler is aware
of data localities and attempts to schedule compute close to the data.
In contrast with GlideinWMS and PanDA, AliEn deploys a pull-based
model~\cite{Saiz:2003:alien}.
\aznote{Need ``why''/justification for pull-based model.}

DIRAC~\cite{1742-6596-219-6-062049} is another comprehensive workload
management system built on top of \pilots.
Akin to PanDA and AliEn, it supports the management of data, which can be placed in
different kinds of storage elements (e.\,g.\ based on SRM).

\mrnote{Do we really want to start comparing/contrasting these certain PJs
that much? This was just the history - but these last 3 paras seems like they
are more about a comparison than just a history}

Another interesting \pilot that is used in the LHC context is
Co-Pilot~\cite{copilot-tr}.
Co-Pilot serves as an integration point between different grid \pilotjob systems
(such as AliEn and PanDA) and clouds.
Co-Pilot is based more on the actual submission of jobs and is limited in its
user-level controllability and allowance of application-level programming.

In addition to the \pilotjob systems developed around the LHC
experiment, several other systems emerged. GWPilot~\cite{gwpilot} is a
\pilot system that is based on the GridWay meta-scheduler. GWPilot, in
particular, emphasizes its multi-user support and the support for
standards, such as DRMAA, OGSA-BES and JSDL. \jhanote{will need
  references to these obscure acronyms! Furthermore, must check that
  the reference to GWPilot is kosher. No bacon allowed.}
\mrnote{It seems a little awkward that we say several others
have emerged, then we say gwpilot is one. but then we go
into talking in next para about scientific workflows + pjs. 
Several makes it sound like we have a bunch more to discuss}
% \msnote{In what way are these systems intrinsically pilot based? I think for
% example with Pegasus, it can make use of pilots, but uses it as just yet
% another "job submission backend". In this way, any system that creates "task"
% could be adopted to submit to a pilot-based backend}


% Many higher-level tools and frameworks, such as workflow,
% visualization or data analytics systems, utilize \pilotjob systems to
% manage their computational workload. In general, two approaches exist:
% (i) the framework vertically integrates with a custom \pilotjob
% implementation (e.\,g.\ Swift/Coaster) or (ii) it re-uses a general
% purpose PJ system (e.\,g\ Pegasus/Condor-G). In case (i), the PJ
% system is also often exposed as stand-alone, multi-purpose \pilotjob systems.

In the context of scientific workflows, \pilotjob systems have proven
an effective tool for managing the workload.  For the Pegasus project,
the Corral system~\cite{Rynge:2011:EUG:2116259.2116599} was developed
as a front-end to GlideinWMS.  In contrast to GlideinWMS, Corral
provides more explicit control over the placement and start of \pilots
to the end-user.
% In contrast to GlideinWMS, Corral-Glide-Ins are run using
% the credential of the user and not a VO credential.
Corral has been developed to support the requirements of the Pegasus workflow
system in particular to optimize the placements of \pilots with respect to
their workload.
% Workflow task clustering with Pegasus~\cite{Singh:2008:WTC:1341811.1341822}.


SWIFT~\cite{Wilde2011} is a scripting language designed for expressing
abstract workflows and computations.
The language provides among many things capabilities for executing external
application as well as the implicit management of data flows between
application tasks.
% For this purpose, SWIFT formalizes the way that applications can define
% data-dependencies.
% Using so called mappers, these dependencies can be easily extended to files or
% groups of files.
The runtime environment handles the allocation of resources and the spawning of
the compute tasks.
Both data- and execution management capabilities are provided
via abstract interfaces.

The Coaster system~\cite{coasters} has been developed to address the workload
management requirements of Swift by supporting various infrastructures, including
clouds and grids.
Using the Coaster service, one executes a Coaster Master on a head node, and
the Coaster workers run on compute nodes to execute jobs.
In the case of cloud computing, Coasters offers a zero-install feature in which
it deploys itself and installs itself from the head node and onto the virtual
machines without needing any prior installation on the machine.
% \msnote{Where does the virtual machine suddenly come from?}
Coaster relies on a master/worker coordination model.
% communication is implemented using GSI-secured TCP sockets.
SWIFT supports various scheduling mechanisms on top of Coaster, e.\,g.\ a FIFO
and a load-aware scheduler. 

\jhanote{I think the previous paragraph can be highly reduced. The
  next paragraph should be modified to highlight that {\bf
    specialized} pilots have also emerged, namely falkon to support
  many short running jobs on HPC systems. This is an important point
  to make and speaks to the success of the pilot concept}

Further, SWIFT can be used in conjunction with other \pilot systems,
e.\,g.\ Falkon~\cite{1362680}.  Falkon refers to pilots as the so
called provisioner, which are created using the Globus GRAM service.
The provisioner spawns a set of executor processes on the allocated
resources, which are then responsible for managing the execution of
task.  Tasks are submitted via a so called dispatcher service.  Falkon
also utilizes a master-work coordination model, i.\,e.\ the executors
periodically query the dispatcher for new tasks.  Falkon was
specifically engineered for many small tasks and shows high
performance compared to native queuing systems.
% Web services are used for communication\msnote{between?}.

WISDOM~\cite{Ahn:2008:ITR:1444448.1445115,wisdom} is an
application-centric environment for supporting drug discovery.  The
architecture utilizes an agent run as a Grid job to pull tasks from a
central metadata service referred to as AMGA. \jhanote{this
  paragraph/pilot can go. I don't see the new functional feature or
  increased complexiy that WISDOM introduces}

\mrnote{I think this section should kind of conclude with something,
but I'm not sure what}
\aznote{Perhaps ``This evolution of \pilotjobs attests to their 
usefulness across a wide range of deployment environments and
application scenarios, but the divergence in specific
functionality + per-\pilot terminology calls for a standard vocabulary
to assist in understanding the varied approaches and their 
commonalities/differences.}


% A LRMS incorporating a pilot scheme:
% OAR~\cite{oar} is a batch scheduler system for clusters and other
% computing infrastructures. Besides its more traditional batch scheduler
% features, it also has the functionality of \textit{container jobs}. These type
% of jobs allow the execution of jobs within other jobs, effectively making it a
% sub-scheduling mechanism, and thereby making it a batch scheduler with
% \pilotjob capabilities.

% Maybe add netsolve later:
% NetSolve~\cite{Casanova:1995:NNS:898848}

%------------------------------------------------------------------------------
% SECTION 3
%------------------------------------------------------------------------------
\section{Understanding the Landscape: Developing a Vocabulary}
\label{sec:vocab}

The overview presented in \S\ref{sec:history} shows a degree of heterogeneity
both in the functionalities and the vocabulary adopted by different \pilotjob
systems. Implementation details sometimes hide the functional commonalities and
differences among \pilotjobs systems while features and capabilities tend to be
named inconsistently, often with the same terms referring to multiple concepts
or the same concept named in different ways.

This section offers an analysis of the architectural components,
functionalities, and terminology shared by every \pilotjob system.  The goal is
to offer both a conceptual description of an exemplar \pilotjob system and a
well- defined vocabulary. Both will be leveraged in \S\ref{sec:4} and
\S\ref{sec:5} to produce a comparative and critical analysis of diverse
\pilotjob frameworks.

\subsection{Core Functionalities of \pilotjobs}
\label{subsec:vocab_core_functionalities}

All the \pilotjob systems introduced in \S\ref{sec:history} are engineered to
allow for the execution of multiple types of workloads on Distributed Computing
Infrastructures (DCIs) such as, Grids, Clouds or HPC facilities. This is
achieved differently, depending on use cases, design and implementation
choices, but also on the constraints imposed by the peculiarities of the
targeted DCI. Finding common denominators among these systems requires an
analysis along multiple dimensions.

%As seen in \S\ref{ssec:evaluation}
% \pilotjobs can be analyzed according to multiple criteria, both functional and
% non-functional.  \onote{avoid functional v.s. non-functional} In this section,
% the focus is on the former as the latter are adapted by definition to each
% specific pilot implementation.

Each \pilotjob system exhibits architectural, provisioning and execution
characteristics that are integral to the \pilotjob paradigm itself.
However, many \pilotjob systems also exhibit specific or non-integral
characteristics. Discerning and distinguishing characteristics into these
categories, requires isolating and defining a minimal set of concepts and terms
that have to apply to every \pilotjob implementation.

At some level, all \pilotjob systems introduced in \S\ref{sec:history}
leverage a similar architecture to accomplish workload execution. Such
an architecture is based on two separate, but intercommunicating
logical components: a \textbf{ Workload Manager} and a \textbf{Task
  Executor}. The Workload Manager is related to the management of the
workloads and their tasks; the Task Executor to the execution of the
tasks on target resources.  As will be shown in \S\ref{sec:4}, the
implementations of both components significantly vary from system to
system, with one or more elements responsible for specific
functionalities both on application as well as infrastructure
level. Nevertheless, the two logical components can be consistently
distinguished and isolated across different \pilotjob systems.

For example, looking at the core \pilotjob systems introduced in
\S\ref{ssec:history}, the scheduler and dispatcher of Nimrod-G
\jhanote{this is now a stale statement, as Nimrod is not discussed in
  Section\ref{ssec:history}} belong to the logical component
dedicated to workload management, while the job-wrapper belongs to the
component responsible for task execution. Similar examples can be
given for the advanced \pilotjob systems and for Pilot-based
frameworks: GlideinWMS \jhanote{we'll need to be careful to not
  conflate a WMS with a PJ systems. It may be just a matter of naming,
  ie replace GlideWMS with Condor-G, but needs care either way..}
offers an extended set of features when compared to the scheduler and
dispatcher of Nimrod-G, but all those features belong to the same,
well-defined and self-contained logical component --- i.e. the one in
charge of managing the workloads. The same is valid for Corral and
\jhanote{Need to replace SWIFT with Coasters -- SWIFT is a workflow
  language; coasters is the underlying pilot implementation.} SWIFT:
both offer evolved functionalities and interfaces to manage workloads
but, as such, both are part of an architecture where workload
management and task execution belong to two distinct
components. \jhanote{fix previous sentence}\mrnote{I think
this paragraph may need an entire revisit, since comparison 
with Nimrod is no longer an option, swift is not a PJ-system, and
glidein is compared vs nimrod}

The Workload Manager and the Task Executor are separate and
well-defined logical components that communicate and coordinate in
order to exchange tasks, input and output files, and related
data. Communication and coordination are two fundamental
characteristics of every distributed architecture \jhanote{suggestion:
  we should replace architecture with either tool or infrastructure or
  something else...},\mrnote{DCI, maybe?} but we posit that it is not an integral logical
component of \pilotjob systems, nor is any specific communication and
coordination pattern a defining feature of the \pilotjob paradigm.
\mrnote{I am really not sold on the claims made in the above paragraph
and then it is not really helped by the below paragraph.
I can see maybe you don't call C and C a logical component, but I would
like an example of a cross-machine pilot job implementation in which there
is no way to communicate or coordinate with any pilots or tasks.  The exact 
definitions given in task dispatching are actually C and C, just not called that.
I am really uncomfortable with the way it is presented as not integral, and I
think we should table this discussion again.}

For example, as seen in \S\ref{ssec:informal}, the \MW paradigm is
very common among \pilotjobs. Functionally, the Master can be
identified with the Workload Manager, while the Worker with the Task
Executor. However, the \MW paradigm usually implies a specific
distribution of capabilities between Master and Worker, and, in some
cases, also a communication model. Moreover, \MW is applied to a
non-fixed set of capabilities while referring not only to
architectures but also to frameworks and applications\jhanote{fix
  sentence}. This flexibility makes \MW a viable choice for
implementing \pilotjobs but not one of its defining characteristics.
\jhanote{the paragraph started talking about C\&C not being integral,
  but then focusses/ends on M-W}

The described logical components of \pilotjob systems support flexible
execution strategies along with improved performance in the task
submission process when compared to scheduling tasks directly on
DCIs.
The tasks of a workload are bound to one or more Task Executor
and, as such, to a pilot, without having to be directly scheduled on
the DCI's job management system.  Furthermore, the binding of tasks to
a pilot might happen before or after the pilot has been assigned a
portion of the DCI's resources, and it can last for the whole time the
pilot can hold on its resources. This execution strategy requires
dedicated functionalities both for performing the binding of tasks to
pilots but also for the provisioning of pilots so that they can be
assigned portions of the DCI's resources. \mrnote{This paragraph
is making it somewhat difficult to understand. A single super
computer or machine is not a 'distributed computing infrastructure.'
Therefore, the assumption when I am reading this is that you may
be referring to a grid. But are you really talking about a machine? 
Because if you're talking about a grid, then you know, submitting to the DCI
would mean submitting to a grid-wide scheduler, not a 
machine-specific scheduler. It should be clarified}

The minimal set of functionalities that needs to be implemented by a
\pilotjob system involves \textbf{\pilot provisioning} and
\textbf{task dispatching}.  \jhanote{need to justify the above
  sentence.}\mrnote{Maybe we can say that we will now motivate
  why we believe the two core functionalities are..., etc. I agree this sounds
  like a strong claim, maybe the conclusion of the paragraph, not the lead}
  \pilot provisioning is a core functionality of every
\pilot system because it is essential for the creation of resource
overlays. This type of overlay allows for tasks to utilize resources
without directly depending on the capabilities exposed by the targeted
DCI. Pilots are scheduled to the DCI resources by means of the DCI
capabilities but, once scheduled and then run, pilots make those
resources directly available for the execution of the tasks of a
workload. \mrnote{This is a much better and explicit use of DCI} 
Thanks to resource overlays, \pilot systems can also
implement functionalities tailored to the requirements of specific
user communities [cit, cit, cit].

The procedures and mechanisms to provision pilots depend on the capabilities
exposed by the targeted DCI and on the implementation of the considered \pilot
system. Typically, for a DCI adopting queues, batch and scheduling systems,
provisioning a pilot involves it being submitted as a job. Conversely, for
infrastructures that do not adopt a job-based approach, a pilot would
be executed by means of other types of logical container as, for example, the
one implemented by means of a Virtual Machine (VM) on so-called
Infrastructures as a Service (IaaS). \mrnote{Should this be a citation?
It just sounds weird when it says ``the one''}

Once pilots are bound to DCI resources, tasks need to be dispatched to those
pilots for execution. Thanks to resource overlays, task dispatching does not
depend on other functionalities provided by the DCI and can be implemented
within the boundaries of the \pilot systems. This independence allows to shift
the control of tasks of a workload directly and exclusively to the \pilot
system, a distinguishing characteristic of such systems.

Data management may have an important role within \pilot systems. For example,
functionalities can be provided to support data staging for task execution or
for managing task-related data according to the capabilities offered by
specific DCI. Nonetheless, \pilot systems can be devised in which tasks do not
require data management because they (i) do not necessitate input files, (ii)
do not produce output files, (iii) data is already locally available or (iv)
data management is left to the application itself. As such, data management
should not be considered a necessary functionality of \pilot systems even when
present in many of them.

In the following subsection, a minimal set of terms related to the
capabilities just described is defined.

\subsection{Terms and Definitions}
\label{subsec:vocab_terms_and_definitions}

\begin{table*}
 \centering
 \begin{tabular}{|l|l|l|}
  \hline
    \textbf{Terms} & \textbf{Functionality} & \textbf{Logical Component} \\ 
  \hline
  \hline
    \textbf{Workload} & Task Dispatching & Workload Manager \\
  \hline
    \textbf{Task} & Task Dispatching & Task Executor \\
  \hline
    \textbf{Resource} & Pilot Provisioning & Workload Manager \\
  \hline
    \textbf{Infrastructure} & Pilot Provisioning & Workload Manager \\
  \hline
    \textbf{Job} & Pilot Provisioning & Workload Manager \\
  \hline
    \textbf{Pilot} & Pilot Provisioning & Workload Manager \\
  \hline
    \textbf{Multi-level scheduling} & Task Dispatching & Task Executor \\
  \hline
    \textbf{Early binding} & Pilot Provisioning & Workload Manager \\
  \hline
    \textbf{Late binding} & Pilot Provisioning & Workload Manager \\
  \hline
 \end{tabular}
 \caption{\textbf{Mapping of the core terminology of \pilot systems into 
  the functionalities and logical components described in 
  \S\ref{subsec:vocab_core_functionalities}.}\up}
 \label{table:terminology}
\end{table*}
\aznote{Suggest reordering introduction of terms from least to most
complex to avoid defining a term using a word which is defined later.
Task, workload, resource, application?} 

The name `\pilotjob' indicates the primary role played by the concepts of
`pilot' and `job' in this type of system. The definition of both concepts is
context-dependent and several other terms needs to be clarified in order to
offer a coherent terminology. Both `job' and `pilot' need to be understood in
the context of DCIs, the infrastructures where \pilotjobs systems are
provisioned\mrnote{Might not need the descriptive clause. We've covered
that already above.} DCIs offer compute, storage, and network resources and
\pilotjobs allow for the users to utilize those resources to execute the tasks
of a workload.

\begin{description}

\item[Application.] Responsible for the definition of the workload and consumer
of the \pilotjob system.

\item[Workload.] A set of tasks, possibly correlated, that are instrumental to
the application to achieve its goal.

\mrnote{We should not use the term workload, as in the definition
of Application, before it is defined. But then application is used circularly
around to itself... not sure how to fix that, but we can't expect someone
to know what 'responsible for the definition of the workload' means when 
they don't know what a workload is}

\item[Task.] A set of operations specified by the application, that are encoded
into one or more programs to be executed on a DCI. 

\mrnote{My issue here is, a single workload 
is a set of tasks. An application, by these definitions, should be comprised
of a possible series of workloads, even though workload is used in the singular
to define an application. If an application is *just* a single workload, then the use of the word
Application is extraneous (strictly going by the definition given above), 
because then you can just use the term 'Workload' to say the 
same thing. Note that you can salvage the definition above, if an application is not defined
as simply a workload definition - i.e. in our asyncre python script, our single workload is really
just the AMBER tasks, but it is considered an 'Application' (in this case) due to the fact
that there's other logic in it besides just the workload definition.  If you want to define a workload
as always singular to an application and just consider the tasks of the workflow to change, 
then in this case, you still need to strengthen the definition - right now, I can think
an application is simply the AMBER tasks and the job launch function which hands
a CU to BigJob. The definition leaves no room for application-level logic, which are
consumers of the task output, etc. On the note of consumers, I find the use
of the phrase 'pilot job consumer',
when not previously discussed or defined in this manner, unclear. At this point, 
you haven't defined what it means to be a Pilot Job system consumer. It is unclear
in what way an Application 'consumes' PJ resources (by definition, not by my knowledge)}

\mrnote{Also, the use of the word program - is this all-encompassing, i.e. do you mean
program binaries and programs? Or do you consider a binary a program? You know,
if you consider a program, a series of instructions for a computer. It's just the word
is a little iffy. I would adopt what BigJob uses which is 'application kernel.' or say
programs and/or binaries}

\item[Resource.] Finite, typed and physical quantity utilized when
  executing one or more workloads. Compute cores, data storage space,
  or bandwidth are all examples of resources commonly utilized by
  running workloads.

\item[Infrastructure.] Structured set of resources, possibly
  geographically and institutionally separated from the users
  utilizing those resources to execute one or more workloads. [cit,
    cit] Infrastructures can be logically partitioned, with a direct
  or indirect mapping onto individual pools of hardware. Commonly,
  Infrastructure can be used also to indicate a DCI. \jhanote{I think
    this entry should be removed from table 1, just like application
    does not find a mention in table 1?}

\end{description}

As seen in \S\ref{ssec:informal}, most of the DCIs where \pilotjobs systems
are executed utilize `queues', `batch systems' and `schedulers'. In such DCIs,
jobs are scheduled and then executed by a batch system.

\begin{description}

\item[Job.] A container for one or more programs, a description of their
  properties and indications on how they should be executed.
\end{description}

In the context of a \pilotjob system, jobs and tasks are functionally analogous
but qualitatively different; it is therefore relevant to highlight their
distinction. Functionally, both jobs and tasks are containers --- i.e. a set of
programs with metadata --- but the term `task' is used when reasoning about
workloads while `job' is used in relation to a specific type of infrastructure
where such a container can be executed. Accordingly, tasks are considered as
the functional units of a workload, while jobs as a way to execute programs on
a given infrastructure. It should be noted that the two terms can be used
interchangeably when considered outside the context of \pilotjob systems.
Workloads are encoded into jobs when they have to be directly executed on
infrastructures that support or require that type of container.

The capabilities exposed by the job submission system of the target
infrastructure determine the submission process of pilots: pilots are programs
with specific capabilities and are submitted as jobs on the type of DCIs just
described \mrnote{I'm uncomfortable with the word programs again, especially
for a 'pilot'. Later we say the term pilot = the term agent = the term pilotjob. Not
sure if I can agree with that - the PilotJob (or Pilot) is just a job (or container)
you submit to the queue.
It's not a running program. Just a job, traditional job script. The 'agent' in this
case is the 'program' that becomes active when the 'job' has become
active in the batch queue. 
Remember that earlier we define a job and a task as being two
separate things (this is at ``CU'' level). But we lump them
together here by saying that the job (Pilot) is the task (program / set of
operations that will be performed). Later, binding is discussed. Early
binding is defined as binding of tasks to inactive pilot. So, if pilot = program,
the definition is binding tasks to an inactive program?}. 
In such a context, schedulers, batch processing and queuing define
the practical boundaries for how and when the pilots can be provisioned on a
given DCI.

Showing that the way in which a pilot is provisioned depends on the
capabilities exposed by the target DCI illustrates the limits of choosing the
term `job'. The use of that term is due to a historical contingency, viz., the
targeting of a specific class of DCIs in which the term `job' was --- and still
is --- meaningful. Nonetheless, with the development of new types of DCI, the
term `job' has become too restrictive, a situation that can lead to
terminological and conceptual confusion, and to the use of synonyms for both
the terms `pilot' and `pilot-job'.

The term `pilot' is often associated with that of `placeholder' so to
emphasize the two distinctive capabilities that every pilot system has to
implement: acquiring a set of resources by running on a DCI, and executing
tasks that will utilize those resources. A pilot is a 'placeholder' because it
holds portion of the DCI resources for a user or a group of users, depending
on implementation details, to gain exclusive control over the binding and
execution of a workload.

\begin{description}
\item[Pilot.] A resource placeholder running on a given infrastructure
  and capable of executing pushed or pulled tasks while managing
  data.
\end{description}

From a terminology point of view, the term `pilot' as defined here is
named differently across multiple \pilotjob systems. Depending upon
context, in addition to the term `placeholder', pilot is also named
`agent' and, in some cases, `\pilotjob' [cit]. All these terms should be
considered synonyms and their use to indicate the same concept is a clear
indication that a minimal and consistent vocabulary is needed when reasoning
about multiple \pilotjob systems.

Once one or more pilots become available through a \pilotjob system on the
targeted infrastructures, users can start to execute their tasks without
having to deal with the job submission system of that infrastructure.
How tasks are assigned to pilots is a matter of implementation. For example, a
dedicated scheduler could be adopted, or tasks might be directly assigned to a
pilot by the user. 

The simplification obtained by circumventing the job submission system of the
DCI is one of the main reasons for the success of the \pilotjob systems. The
overhead and lack of control imposed by a centralized job management system
shared among multiple users are bypassed, allowing for a faster and more
reliable execution of workloads.

The \pilotjob systems are said to implement multi-level scheduling because the
assignment of resources to tasks happens in at least two stages. A portion of
the resources of an infrastructure are first bound to a pilot and then to one
or more tasks for consumption. This is an important feature of \pilotjob
systems because it allows for a task to be bound to a pilot before it is in
turn bound to the resources.

The binding of tasks to pilots depends on the state of the pilot. A pilot is
inactive until it is executed on a DCI, active after that. Early binding
indicates the binding of a task to an inactive pilot; late binding the binding
of a task to an active pilot. Early binding is useful to increase the
flexibility with which pilots are deployed. By knowing in advance the
properties of the tasks that are bound to a pilot, specific deployment
decisions can be made for that pilot. Late binding is critical in assuring high
throughput by allowing for tasks to be executed without waiting in a queue or
waiting for a specific container - for example a job or a VM - to be
instantiated.

Binding pilots to their resources and tasks to pilots depends on placement
choices. As such, the binding process in itself is often an instance of
scheduling. Pilot systems implement multi-level scheduling because they require
the scheduling of two types of entities - pilots and tasks - in a distinct
chronological order. \jhanote{merge the previous sentence with the   paragraph
beginning with The \pilotjob systems are said to implement multi-level
scheduling because..} \aznote{Agreed, need to not give two different
definitons of ``multi-level scheduling'' as well?}

\begin{description}

\item[Early binding.] Binding one or more tasks to an inactive pilot.

\item[Late binding.] Binding one or more tasks to an active pilot.

\item[Multi-level scheduling.] Scheduling pilots onto resources and tasks onto
active or inactive pilots in distinct chronological order.

\end{description}

Depending on the specific capabilities implemented in the workload management
component, some \pilotjob systems allow for pilots to be specified by taking
into consideration the properties of the task of a workload. This type of
specification process should not be confused with early binding as the latter
requires for a pilot to have been already specified but not yet bound to a
resource.

%------------------------------------------------------------------------------
% SECTION 4
%------------------------------------------------------------------------------
\section{Pilot-Job Systems Implementations}\label{sec:4}

Section \S\ref{sec:vocab} offered two main contributions: a minimal description
of the logical components and the functionalities of \pilotjob systems, and a
well-defined core terminology to support reasoning about such systems. The
former sets the necessary and sufficient requirements for a distributed system
to be a \pilotjob system, while the latter enables consistency when referring
to different \pilotjob systems. Both these contributions are leveraged in this
Section in order to review critically a set of relevant \pilotjob
implementations.

% \jhanote{More than twofold? I think we are missing the contributions
%   of section 4.1 and 4.2}\mtnote{I would suggest to develop further
%   the rest of the Section and then come back and refine this one.}

The goal of this Section is twofold.  Initially, the set of
functionalities presented in \S\ref{sec:vocab} are used as the basis
to infer a set of core implementation properties. A set of auxiliary
properties is also defined when useful for a critical comparison among
different \pilotjob implementations. Subsequently, several \pilotjob
implementations are analyzed and then clustered around the properties
previously defined. In this way, insight is offered about how to
choose a \pilotjob system based on functional requirements, how
\pilotjob systems are designed and engineered, and the theoretical
properties that underly such systems.

% \jhanote{Matteo: I think the order of 4 differs from the message conveyed in
% the previous paragraph. Also, I think the clustering is the reverse of what is
% suggested, viz., the properties are defined/derived, and then pilot-jobs
% systems are clustered around the properties and not vice-versa. agreed?}
% \mtnote{Any better?}

% Now that we, build up a vocabulary, distinguished logical component,
% described core functionalities and narrowed down definitions of pilot jobs,
% we can use use this acquired knowledge to analyse a certain set of \pilotjob
% frameworks.

% Note that above properties were derived in the context of coming up with a
% minimum framework for pilot frameworks. Given that, there are more properties
% that are distinctive for the various frameworks, eventhough they might not be
% distinctive for pilot-jobs per se.

% The set of frameworks we use for the analysis is chosen because of criteria x
% and y.

% Logical: Workload Manager / Task Executor
% Functionalities: Pilot Provisioning / Task dispatching

\subsection{Core and Auxiliary Properties}

This Section discusses the properties of diverse implementations of a \pilotjob
system. Two sets of properties are introduced: Core and auxiliary.
\textit{Core} properties are common to all \pilotjob implementations while the
auxiliaries characterize specific \pilotjob systems. Therefore, auxiliary
properties are not shared among all \pilotjob implementations.

As shown in Table \ref{table:core_properties}, the set of Core Properties is
derived consistently with the set of functionalities presented in
\S\ref{sec:vocab} - Pilot Provisioning, Task Dispatching, and Task Execution.
As such, Core Properties are both necessary and sufficient for an
implementation of a distributed system to be classified as a \pilotjob. On the
contrary, auxiliary properties are not defining of a \pilotjob system but they
are implementation details that further characterize a \pilotjob system.  The
set of auxillary properties we discuss is not closed, i.e., the complete set of
auxillary properties includes, but is not limited to the set of auxillary
properties we discuss.

% \jhanote{Give reader a reminder that ``the set'' of core properties
%   are both necessary and sufficient. Also, maybe this is intended or
%   implied, but it will help to make explicit that auxiliary properties
%   are not a closed set of properties, unlike core}\mtnote{First go at
%   it. Not sure what you mean exactly with 'closed' in this context.}
% \jhanote{Matteo, please see last sentence. Make sense?}\mtnote{Yep.}
\mrnote{Resource Dependences is used a lot here, but I am pretty sure
dependence cannot be pluralized like that. It might be dependencies?
PS Merriam-Webster agrees}

\begin{table}
\centering
 \begin{tabular}{|l|l|}
  \hline
    \textbf{Core Propriety} & \textbf{Functionality} \\ \hline
  \hline
    Pilot Characteristics & Pilot Provisioning \\
  \hline
    Resource Dependences & Pilot Provisioning \\
  \hline
    Workload Semantics & Task Dispatching \\
  \hline
    Task Binding Characteristics & Task Dispatching \\
  \hline
    Deployment Strategies & Task Execution \\ \hline
 \end{tabular}
 \caption{\textbf{Mapping of the Core Properties of \pilotjob system 
                  implementations into the functionalities described in 
                  \S\ref{subsec:vocab_core_functionalities}.}\up}
 \label{table:core_properties}
\end{table}

Implementations of Pilot Provisioning are analyzed by focusing on two specific
properties: the requirements and dependencies between the \pilotjob system and
the underlying distributed resources; and the type of resources the pilot
system exposes in terms of computing, data and networking. For example, in
order to schedule a pilot onto a specific resource, the \pilotjob system will
need to know what type of container to use (e.g. pilot, virtual machine), what
type of scheduler the resource exposes, but also what kind of functionalities
will be available on the nodes of the resource in terms of compute, data and
networking.

% \jhanote{please elaborate on the previous sentence, maybe
%   with an example?}\mtnote{Done.}

Two properties are also used to analyze the implementations of Task
Dispatching: The semantics of workloads, and how the tasks of a given workload
can be bound to single or multiple pilots. Semantically, a workload description
contains all the information necessary for it to be dispatched to the
appropriate resource. For example, information related to both space and time
should be available when deciding how many resources of a specific type should
be used to execute the given workload but also for how long such resources
should be available. Executing a workload requires for its tasks to be bound to
the resources. Both the temporal and spatial dimensions of the binding
operations are relevant for the implementation of Task Dispatching. Depending
on the concurrency of a given workload, tasks could be dispatched to one or
more pilots for an efficient execution. Furthermore, tasks could be bound to
pilots before or after its instantiation, depending on resource availability
and scheduling decisions.

% \jhanote{I observe, we do not discuss why ``task binding
%   characteristics''. In conjunction, with discussion in 4.1.1, I
%   believe reader is left unsure why this is fit for inclusion as a
%   core property. Please see specific comment in 4.1.1 under task
%   binding characteristics}\mtnote{We had a discussion of binding but
%   very implicit and limited to the spatial dimension. I have now
%   extended the explanation adding the temporal dimension and
%   mentioning binding explicitly.}

Finally, implementations of Task Execution are analyzed by reviewing different
strategies for task scheduling and by describing how \pilotjob systems support
task execution. \pilotjob implementations may offer multiple scheduling
strategies depending on varying factors related to the nature of the workload,
the state of the resources, or the capabilities exposed by the underlying
middleware. Furthermore, \pilotjob systems often are responsible for setting up
the execution environment for the tasks of the given workload. While each task
can be seen as a self-contained and self-sufficient unit with a kernel ready to
be executed on the underlying architecture, often tasks require their
environment to be set up so that some libraries, data or accessory programs
are made available.

Several auxiliary properties play a fundamental role in distinguishing
among \pilotjob systems implementations, as well as address, set and
provide constraints on their usability.  Programming and user
interfaces; interoperability across differing middleware and other
\pilotjob systems; multitenancy; strategies and abstractions for data
management; security including policies alongside authentication and
authorization; support for multiple usage modes like HPC or HTC; or
robustness in terms of fault-tolerance and high-availability; are all
examples of properties that might characterize a \pilotjob
implementation but in of themselves, would not distinguish a \pilotjob
as a unique system.

% \mtnote{I have been assuming that a \pilotjob is always a type of distributed
% system. Do you agree?}. \jhanote{This needs discussion. It is definitely a
% system, and definitely a software system. But is a distributed tool or a
% distributed service always also a distributed system?} \mtnote{I guess we need
% to pin down a proper definition of distributed system. Do you have a preferred
% one?} \mtnote{After a shared discussion, we decided to change the paragraph so
% that the issue does not arise. The issue itself can be addressed by pinning
% down the level of abstraction and defining 'distributed' at that level but, as
% such, it will also offer a relative definition of the class of system to which
% the pilot belongs and, as such, not a very interesting one.}

Both core and auxiliary properties have a direct impact on the multiple use
cases for which \pilotjob systems are currently engineered and deployed. For
example, while every \pilotjob system offers the opportunity to schedule the
tasks of a workload on a pilot, the degree of support of specific workloads
varies vastly across implementations. Furthermore, some \pilotjob systems
support Virtual Organizations and running tasks from multiple users on a single
pilot while others support jobs leveraging a Message Passing Interface (MPI).
Analogously, every \pilotjob systems, support the execution of one or more type
of workload but they differ when considering execution modalities that maximize
throughput (HTC), computing (HPC) or container-based high scalability (Cloud).

\mtnote{Are we going to introduce use cases in   the Introduction or in Section
2? Once decided, I would add an   internal reference after mentioning the use
cases.}. \jhanote{Do we   need to distinguish usability versus use-
case?}\mtnote{I did not think about this distinction. I meant 'use cases'
because I think we need some explanation on how these systems are used when we
will show that some are tailored around a specific community while others are
general-purpose. It should also be useful when arguing about the limitations of
the \pilotjob paradigm - we do not want to give the impression that \pilotjobs
are the solution for every possible user.} \jhanote{I need to think, but there
is the issue of internal versus external constraints on the   usability.
Auxiliary properties seem to me be about responding to   external factors; core
properties seem to determine usability from   internal properties. I do not
know if this additional qualifications   or helps, but needs thinking and
discussion}\mtnote{I need to think too, happy to discuss about this
one.}\mtnote{We decided to integrate, discourselvely, use cases references in
S2 when \pilotjob systems are introduced.}

% \jhanote{(i) A workflow-based application and MPI are   not mutually exclusive.
% (ii) Agree/Disagree: A workflow-based   application by the time it is supported
% by a \pilotjob system is   just a workload, and semantics of the workflow lost?
% If agree, then   we should use another example}\mtnote{Done.} 

% \mtnote{We need now to 'justify' both types of proprieties. Why do we choose
% these proprieties? I think we should find this justification internally, i.e.
% by showing that each propriety can be brought back to those - more general -
% introduced in S3. This could be done by adding a table that maps the core
% proprieties to the functionalities of Table 1 in S3. The idea here is that the
% proprieties - BTW, we may want to simplify naming here and rethink about using
% the term `propriety' altogether - in S4 are `contained' in the functionalities
% listed in S3. With `contained' I mean that they are refinement of those
% functionalities at a lower level of abstraction and, as such, in 1 to N
% relationship so that multiple proprieties in S4 can be brought back to a single
% functionality in S3. This mapping would be the internal justification of the
% choice of the core proprieties. An initial example of the Table I am
% speaking about in Table \ref{table:core_properties}.}

% \mtnote{What about about the justification of the auxiliary proprieties? Well,
% here I need to think a bit more about it. At the moment, I am cogitating
% whether those proprieties could be seen as transversal to the functionalities
% listed in S3, a byproduct of lowering the level of abstraction but still, in
% principle, in the scope of S3 `model'. For example, the implementation language
% cut across all the core functionalities - assuming a single implementation
% language - and so does multi-user support as it maps on both task provisioning
% and pilot dispatching.}

\mtnote{For S3, the TODO list is: 1. Argue explicitly that the offered `model'
is sufficient and necessary in order to discriminate between \pilotjob and
not-\pilotjob systems; 2. Extend the `model' so to include a logical element
and one or more functionalities related to Task Execution, and possibly another
one relative to overlay enacting as distinguished from task dispatching; 3.
Clean up Table 1 removing 'Infrastructure'.}

% \mtnote{As a general note, I see the current Subsections 4.1 and 4.2 as a
% brainstorming for refining the current Subsections 4.5 and 4.6. I would not
% want to rewrite them from scratch as I think there is a lot of substance in
% them as they are at the moment.}

\subsubsection{Core properties}

\mrnote{I didn't have a problem with the below paragraph
until I tried to revise 4.2. As it turns out, Pilot characteristics
are not core properties of the PJ system but actually of the 
Pilots themselves. Like, number of cores... how would I answer that?
The number of cores you can request is system-limited. Lifespan,
again this depends on wall clock time given to the job. It is hard
to say okay a core property of BigJob is the characteristics of 
its Pilots - but I can't go on to say BigJob pilots have X size,
Y lifespan, etc. because its individual pilot-dependent, and doesn't
have anything to do with BigJob really}

\begin{itemize}

\item \textbf{Pilot Characteristics}: In \S\ref{sec:vocabulary} pilots
  have been defined as placeholders for resources. As such, the
  characteristics of each pilot depends upon the type and capabilities
  of the resource it exposes.  Pilots usually expose computing
  resources but, depending on the capabilities offered by the
  infrastructure where the pilot is instantiated, they might also
  expose data and network capabilities. Within the domain of each type
  of resource and infrastructure, some of the typical characteristics
  of pilots are: Size (e.g. number of cores), lifespan,
  intercommunication (e.g. low-latency or inter-domain), computing
  platforms (e.g. x86, Cray, or CUDA), file systems (e.g. local,
  shared, or distributed). 

  % \jhanote{Opening clause, needs revision.  Sorry I can't proffer
  %   something, as I'm not sure of the intent}\jhanote{I'm a bit
  %   confused between resource and infrastructure semantics now. See
  %   general remark at the end of Core Properties. Does that help
  %   refine this paragraph?}\mtnote{I tried to make it more
  %   consistent with the definitions we gave in S3.}
\mrnote{dependences vs dependencies}
\item \textbf{Resource Dependences}: The provisioning of pilots
  depends on how the \pilotjob system interfaces with one or more
  targeted infrastructure(s). In this context, the degree of coupling
  between the \pilotjob system and the infrastructure can vary,
  depending on how the \pilotjob system is deployed, how much it is
  integrated with the middleware used within the targeted
  infrastructure, whether the \pilotjob system is interoperable across
  multiple types of middleware, and how much information the \pilotjob
  system needs about the states and capabilities of the targeted
  infrastructure.

\item \textbf{Coordination and Communication}. \mtnote{To be decided
  whether we need it. I think we do.}\jhanote{to be
  discussed/determined after 20th November based upon feedback and
  analysis from 4.2 update/refinements}\mrnote{Agree with Mark}

\item \textbf{Workload Semantics}: The tasks of a workload are
  dispatched to pilots depending on their semantics. Specifically,
  dispatching decisions depends on the relationship held by a task
  with the other tasks belonging to the workload, the affinity they
  require between data and compute resources, and the type of
  capabilities they require in order to be executed.  \pilotjob
  implementations support a varying degree of semantic richness for
  the workload and its tasks.

\item \textbf{Task Binding Characteristics}: One of the core
  functionalities implied by Task Dispatching is the binding of tasks
  to pilots. Without such capability, it would not be possible to know
  where to dispatch tasks, pilots could not be used to run tasks and,
  as such, the whole \pilotjob system would not be usable. As seen in
  \S\ref{sec:vocab}, \pilotjob systems may allow for two main types of
  binding between tasks and pilots: early binding and late
  binding. \pilotjob implementations differ in whether and how they
  support these two types of binding. Specifically, while there might
  be implementations that only support a single type of binding
  behavior, they might also differ in whether they allow for the users
  to control directly what type of binding is performed, and in
  whether both types of binding are available on an heterogeneous pool
  of resources.

% \jhanote{I agree with the discussion above, but there is case to be
%   made that at some stage, we will have to argue why this is a core
%   and not an auxiliary property, as one could just reduce this to an
%   issue of control/implicit versus explicit?}\mtnote{I added a first
%   try at an explanation. We can stop to the first sentence if you
%   think it is sufficient otherwise we can leave also the following.}
\mrnote{Deployment strategies sounds like you mean how can
the pilot job system be deployed, but it turns out it is how the environment
required by the task can be created.}
\item \textbf{Deployment Strategies}: Once the tasks are dispatched to
  a pilot, their execution may require for a specific environment to
  be set up.  \pilotjob implementations differ in whether and how they
  offer such a capability. \pilotjob implementations may adopt
  dedicated components for managing execution environments, or they
  may relay an ad hoc configuration of the pilots. Furthermore,
  execution environments can be of varying complexity, depending on
  whether the \pilotjob implementation allows for data retrieval,
  dedicated software and library installations, communication and
  coordination among multiple execution environment and, in case,
  pilots.

\end{itemize}

% \jhanote{General comment: I can't remember if we distinguish
%   explicitly resources: compute, data and/or network; infrastructure:
%   specific physical realization of a resource. If not, we should
%   mention upstream}\mtnote{Yes, we define them along these lines in S3.}

\subsubsection{Auxiliary properties}

\begin{itemize}

\item \textbf{Architecture}: \pilotjob systems may be implemented by
  means of different type of architectures (e.g service-oriented,
  client-server, or peer- to-peer). For example, it is conceivable
  that architectural choices influence if not preclude certain
  deployment strategies. The analysis and comparison of architectural
  choices is limited to the trade-offs implied by such a choice,
  especially when considering how they affect the Core Properties
  listed in the previous Subsection.

% \jhanote{A simple/quick example would help appreciate the latter
%   sentence; currently not sure. What about the following?: ``For
%   example, it is conceivable that architectural choices/decisions
%   influence if not preclude certain deployment
%   strategies.''}\mtnote{Thank you.  Integrated into the text.}

% \jhanote{sytlistic/consistency: either we use ``Core Properties'' or
%   ``core properties''. Can ignore for now, but please converge/keep
%   in mind.}\mtnote{Done.}

\item \textbf{Interfaces}: The implementations of \pilotjob systems
  may present several types of interfaces. For example, there are
  interface considerations between the main components of the
  \pilotjob system, between the application and pilot layer, between
  end users and the \pilotjob system, and for one or more programming
  languages.

\item \textbf{Interoperability}: Two types of interoperability are
  relevant when analyzing different \pilotjob implementations:
  Interoperability with multiple type of resources and
  interoperability across diverse \pilotjob systems. The former allows
  for the \pilotjob to provision pilots and execute workloads on
  different type of resources and systems (e.g. HTC, HPC, Cloud but
  also Condor, LSF, Slurm or Torque), while the latter becomes
  relevant when considering a landscape where multiple \pilotjob
  implementations are available with diversified characteristics and
  capabilities.\mrnote{Again, we are in the section labeled 'Auxiliary 
  Properties' for PilotJob Systems. We are *not* talking about a layer
  above that. You cannot have interoperability between PJ systems
  at the level of the PJ system itself, otherwise it would be considered
  a PilotJob Overlay of some sort and not an individual PJ system}

\item \textbf{Multitenancy}: A \pilotjob system may offer multitenancy
  at both system and local level. When offered at system level,
  multiple users are allow to utilize the same instance of a \pilotjob
  system, while when available at local level, multiple users may use
  the same pilot instance or any other component implemented within the 
  \pilotjob system.

  % \jhanote{Editorial/Minor note: I think I know what you mean by
  %   ``service within the PJS'', but it is not explained or detailed
  %   somewhere. Maybe use vocabulary/terminology already
  %   defined?}\mtnote{Thanks. I went for components alone because
  %   services would still be components of a \pilotjob
  %   system. Service are a way to implement its components.}

\item \textbf{Robustness}: Used to identify those properties that
  contribute towards the resilience and the reliability of a \pilotjob
  implementation. In this Section, the analysis focuses on
  fault-tolerance, high-availability and state persistence. These
  properties are considered indicators of both the maturity of the
  development stage of the \pilotjob implementation, and the type of
  support offered to the paradigmatic use cases introduced in Section
  2. 

\item \textbf{Security}: While the properties of \pilotjob
  implementations related to security would require a dedicated
  analysis, we limit the discussion to authentication, authorization
  and policies. The scope of the analysis is further constrained by
  focusing the analysis only on those elements of these properties
  that impact the Core Functionalities as defined in
  \S\ref{sec:vocab}.

\item \textbf{Usage Modes}: As shown in \S\ref{sec:history} and 
\ref{sec:vocab}, \pilotjob systems support many different use cases involving
diversified usage modes - e.g. HPC, HTC and Cloud.

% \item \textit{Interfaces:} How does the \textbf{application} (responsible for
% definition of the \textbf{workload} and consumer of the pilot-job system)
% interact/interface with the pilot-job system? (API / CLI / Service / etc).

% \item \textit{Programming Language:} In the case of an API, what language is bindings are
% available?

% \item \textit{Multi-user support:} Does the pilot job system allows \textbf{\pilots} to
% be used by multiple users?

% \item \textit{Resource/Pilot granularity:} How is the slicing and dicing of a
% resource exposed to the \textbf{pilot} and how much control is there on the
% granularity? How does this effect the suitability for HTC and/or HPC?

% \item \textit{Supported Middleware:} What middleware does the system connect with? Is it
% tightly coupled or flexible/extensible?

% \item \textit{Data management:} Does the framework provide data management
% capabilities or should that be done out of band? If so, what does it offer?

\end{itemize}

% \subsection{Organizing properties in space}

% \mtnote{TODO: Move all this under the following subsection and normalize
% following the revised list and definition of both types of properties.
% Eliminate the use of the term 'Space' if not needed.}

% \subsubsection{Space: Workload Management}
% "How to express the workload."
% \begin{itemize}
% \item Task semantics (Core)
% \item Workload semantics (Core)
% \item Data Management (Auxiliary)
% \end{itemize}

% \subsubsection{Space: Pilot Provisioning}
% "How to create a resource overlay."
% \begin{itemize}
% \item Pilot Instantiation (Core)
% \item Placeholder characteristics (Core)
% \item Resource/Pilot granularity (Auxiliary)
% \end{itemize}

% \subsubsection{Space: Scheduling}
% "How \pilots and \cus are scheduled."
% \begin{itemize}
% \item Binding behavior (Core)
% \item Resource exposure (Core)
% \end{itemize}

% \subsubsection{Space: Ecosystem}
% "How the system interacts with its environment."
% \begin{itemize}
% \item Usage Interface (Auxiliary)
% \item Programming Language (Auxiliary)
% \item Supported Middleware (Auxiliary)
% \item Multi-user support (Auxiliary)
% \end{itemize}


% \subsection{Old section 4 from here}


% Despite the apparent diversity in existing \pilotjobs,
% they can be reduced to a set of common functionality and
% discussed as such via use of the the terminology presented
% in \S\ref{sec:vocab}.
% %has implications
% %for the interoperability of existing and future \pilotjobs, and
% %lends insight into the design and usage of \pilotjob systems.
% % application design to enhance job/data(?)
% %mobility, management, and potential run-time responsiveness via
% %late-binding/etc.


% \textcolor{red}
% {
% \textbf{Ideas 4.1.X structure}
% \begin{itemize}
% \item CHECK OVERLAP WITH 2.1
% \item Why did we pick this specific PJ? - already described in section 2?
% \item What was this PJ designed for (motivates architecture)
% \item What applications are using this PJ (if any?)
% \item (deployment assumptions) What is the typical deployment and usage 
% scenario (e.g., VO. multi-user)
% \item (resource landscape assumptions) What is the resource landscape it 
%  assumes (Clouds, HPC, 'Grid', single machines,... )
% \item Architecture
%   \begin{itemize}
%     \item Components (mapping to 3)
%     \item How do components communicate? (e.g., push/pull propr. TCP 
%      protocol, REST)
%     \item Interface (how is it accessed)
%   \end{itemize}
% \end{itemize}
% }

% %\subsection{Comparing \pilotjobs via a Common Framework \& Vocabulary}

% In light of the common vocabulary discussion in \S\ref{sec:vocab}, a
% representative set of \pilotjobs has been chosen for further analysis.
% Examining these \pilotjobs using the common vocabulary exposes their
% core similarities, and allows a detailed analysis of their
% differences.

% We will see that throughout the evolution of the \pilotjob concept,
% several core concepts have remained stable while new additions enhance
% cross-platform interoperability, ease of usage and deployment, and
% incorporate the ability to schedule data as well as jobs.  These
% additions illustrate the flexibility and power of \pilotjobs, and
% emphasize the need for a common vocabulary/treatment to describe the
% developing landscape.


% % At least one \pilotjob from each categorization in \S\ref{sec:history}
% % is analyzed via a common vocabulary, allowing for a standard
% % ``baseline'' of each \pilotjob categorization to be examined.

% \jhanote{this might go to the section where we begin describing
%   pilotjobs.. also will need an updated description of the methodology
%   employed to choose} ....such that ``exemplars'' were chosen; these
% are \pilotjobs which either laid foundational \pilotjob concept or
% incorporated strides in interoperability, usability, architecture or
% otherwise to advance the understanding and usage of \pilotjobs.

% \jhanote{We need to provide an overview/outline of this section}

% %Representative \pilotjobs from each \pilotjob category
% %in Section~\ref{sec:survey} are chosen and compared within
% %the bounds of that category, illustrating the core similarities
% %between the Pilot-Jobs in a given category.

% % \textit{Finally, we perform
% % a cross-cutting analysis, demonstrating that across the categorical
% % distinctions all approaches remain true ``\pilotjobs'' with regard
% % to their base functionality.  The segmentation between categories
% % is exposed to reveal why \pilotjobs have evolved to have
% % differing capabilities, with the end-goal being a
% % complete understanding of the \pilotjob ecosystem via a common
% % vocabulary.}

% \subsection{Evaluation Criteria}\label{ssec:evaluation}
% \jhanote{I think we agreed to rename this sub-section, but can't
%   remember to what! yikes..}

% \begin{figure}[t]
% 	\centering
% 		\includegraphics[width=0.45\textwidth]{figures/pilot-criteria}
%                 \caption{\jhanote{This is by no means complete.. but
%                     just a way to help start thinking/organizing
%                     criteria}..}
% \end{figure}

% % \jhanote{Andre L: these need elaboration and explanation. A possible
% %   consideration might be whether to remove bullet points and use
% %   regular prose}\alnote{work in progress}

% % \pilotjob systems can be described and compared using a set of
% % functional and non-functional criteria.

% %\alnote{use functional/non-functional properties}

% Section 2 and 3 discussed how \pilotjobs have evolved functionally, as
% well discussed the different ways in which \pilotjobs have been
% used and developed. It follows that there exist multiple aspects and
% attributes of \pilotjobs that need to be kept in consideration when
% trying to determine an appropriate implementation.  Here we try to
% gather the broad set of criteria, most but not all of which follows
% from earlier discussions.

% In order to understand the landscape of \pilotjobs it is important to
% understand the types of workflows and workloads they can support, the
% types of resources they can be used with and how, what middleware
% dependencies they have, the type, level and granularity of scheduling
% control they expose and the algorithms they use, control on the
% workload placement they provide and the
% constraints on the workload they impose. 

% Whereas the answers to some of the above influence the usability and
% usefulness of a \pilotjob, not all of the above attributes influence
% their functional or performance capabilities.  \jhanote{the following
%   might/should go} No rigid classification of the attributes of
% \pilotjobs is possible, as an attribute that influences the
% performance or functional capability of one \pilotjob may not
% influence or even exist for another \pilotjob. Mindful of these
% limitations, we propose the following features that provide a fair
% coverage of the attributes required to understand the usability (which
% correlates with non-functionional properties etc.)  and usefulness
% (performance constraints, functional capabilities etc.)  of any given
% \pilotjob. \jhanote{need consistency: we use attribute,
%   characteristics, features to describe the same thing}

% What use cases (application workloads, MPI/ensembles, workflow types)
% and use modes are supported (natively or via extensions) by the core
% \pilotjob system?  How is the \pilotjob system used deployed, viz.,
% what abstraction and user interfaces are provided to the user (e.g.,
% CLI, web-service, web portal?). Does the \pilotjob system support
% distribution? What middleware dependencies does the \pilotjob system
% have?  What execution strategies are supported? (HTC, HPC, cloud)?
% What type of security contexts are supported?  Once a \pilotjob is
% usable, attributes that influence whether a \pilotjob is useful include
% the scalability (number of tasks, number of distributed resources,
% number of cores that can be marshalled, task throughput, number
% resources/slots) properties of the \pilotjob, interoperability and
% fault tolerance.

% %Does the \pilotjob system provide interoperability support?
% % does the framework support advanced workload management capabilities
% % (e.g. the automatic provisioning of \pilots).

% % \emph{Functional Features:} The functional criteria describe the
% % capabilities of the \pilotjob system: (i) What use cases are
% % targeted by the PJ system? What applications types and usage modes
% % are supported (MPI, ensembles, workflows)?, (ii) What abstraction
% % and user interface is provided to the developer: API, CLI, web
% % service, web portal?, (iii) Does the \pilotjob system support
% % distribution?, (iv) Are there any higher-level tools available for
% % \pilotjob system (e.g. workflow systems, data analysis systems)?,
% % (v) Does the \pilotjob system support data?, (vi) What types of
% % infrastructure are supported (HTC, HPC, cloud)? Does the \pilotjob
% % system provide interoperability support? and (vii) does the
% % framework support advanced workload management capabilities
% % (e.g. the automatic provisioning of \pilots).


% % \emph{Non-Functional Features:} Non-functional features describe the
% % characteristics and behavior of a \pilotjob system and are not
% % directly related to a specific function or capability of the \pilotjob
% % system. Non-functional properties include: 

% % \jhanote{I think performance and scalability should be moved into
% %   functional attributes. I think performance and scalability (as
% %   measured by tps, resources/slots) are in deed capabilities of a
% %   \pilotjob. Also, I think, ``What abstraction and user interface is
% %   provided to the developer: API, CLI, web service, web portal?'', as
% %   well as ``Are there any higher level tools'', should be moved to
% %   non-functional features. If functional and non-functional is not a
% %   good enough classification, then we might further sub-divide
% %   functional attributed; having already discussed core, advanced and
% %   higher-level pilot, I think we are losing the opportunity to connect
% %   to and utilize that discussion}

% \msnote{Formally performance, etc. are non-functional attributes. But
%   performance / scalability is often a reason to use pilot-jobs, so
%   what I think you are asking for is a more prominent place to discuss
%   performance? Does that tell us that functional vs non-func is not a
%   useful distinction here?  (func/non-func normally belongs to
%   requirements, and thats not what we are talking about here
%   anyway}\jhanote{Mark: mostly addressed in discussions, comments
%   (removal) and reworking}

% \jhanote{the following needs to be refined, as it is a very narrow
%   definition of performance} Performance and scalability define the
% response times and size of the workload a user can expect to be
% supported.  Security describes the security mechanisms used by the
% framework, e.\,g.\ for authentication of the user, for securing the
% communication protocol and for sandboxing the application.  Further,
% some \pilotjobs solely support single users, while more complex
% \pilot-based workload managers commonly support multiple users
% (e.\,g.\ using glexec).

% Further non-functional aspects describe the internals of a \pilotjob system,
% e.\,g.: (i) the architecture of the system (layers, sub-systems,
% communication \& coordination, central vs. decentral architectures; push vs.
% pull model, agent-based; number of supported resource types), (ii) resource
% access layer: What low-level infrastructures/middleware are supported? How
% interoperable is the framework (vertically, horizontally)?, (iii) the
% deployment model: hosted service versus tool/library, application vs.
% system-level, (iv) how is the workload managed and scheduled (supported
% algorithms, support for application-level scheduling, data-compute scheduling,
% resource acquisition/release policies, dispatch policies, number of scheduling
% levels) and (v) does the framework have dependencies to other third-party
% components/services?

% \subsection{Comparing \pilot Systems via Classifiers}

% Along with the common framework and vocabulary, we have identified a set of  key
% classifiers along which we discuss the functional and non-functional properties
% of \pilot systems.  While these classifiers are by no means meant to be
% exhaustive or complete, they are a useful tool to highlight and discuss the
% differences in  usage-modes, deployment and capabilities between different
% \pilotjob systems. We divide our classifiers into three functional high-level
% categories: \textit{Resource Interaction}, \textit{Workload Execution and
% Management} and \textit{Data Handling and Management}. In each category we
% define classifiers that give further information about (1) if and how a
% capability is implemented, and (2) if and on which level a capability is
% controllable. A fourth set of classifiers covers the non-functional properties
% of \pilot systems.


% \paragraph{Pilot Provisioning}

% \mtnote{Integrate 4.2/4.3 in this subsection.}

% \begin{itemize}

% \item \textit{Pilot ownership}: Who owns the pilot (job) -- the user or the
% \pilotjob system? Or in other words, in which security context does the pilot
% operate? \msnote{Dont think we have defined "user" by now, so thats ambiguous}
% \msnote{As there will be a wide spectrum of approaches in the various
% implementation, we might just want to call this as general as "security"}
% \aznote{I think this covers potentially multiple concepts: 1) security, as in
% the level of direct access users have to pilots (can they for instance force a
% kill directly?  Modify pilot configurations while it runs?) 2) Deployment, as
% in does the entire pilot-job system operate in user-space or is it part of the
% (admin/privileged) system stack? 3) I would argue that this could relate
% directly to pilot-instantiation... both from a security and deployment level.
% Who can instantiate pilots? What security level do they run on once
% instantiated?  So, maybe a ``security'' item as Mark suggested, but I'm not
% entirely sure if ``deployment'' should be part of this as well...  they do seem
% connected to me though...}

% \item \textit{Supported DCI systems/services}: What are the type of systems
% (middlewares, services) that can be interfaced by the \pilotjob system to
% launch pilots?

% \item \textit{Pilot Granularity (spatial)}: What unit of resource is controlled
% by a single pilot? A core, a node, multiple nodes, a cluster, ...?

% \item \textit{Pilot instantiation (spatial+temporal)}: When are pilots
% instantiated where and which entity (system/user) has control over it at
% which level?

% \end{itemize}


% \paragraph{Workload Management}

% \begin{itemize}

% \item \textit{Workload description}: How are workloads described, e.g.,
% ClassAdd. JSDL...?

% \item \textit{Supported ``logical'' workload types}: What types of workload
% are supported, e.g.      single tasks, workflows, ...? (here ``logical'' as
% opposed to ``physical'' task types in the next category)     \msnote{It seems
% this is a special case of the above item}

% \item \textit{Task-to-pilot assignment}: How are tasks assigned to pilots and
% at which level of     granularity can this be controlled by which entity
% (system/user)?

% \end{itemize}

% \paragraph{Task Execution}

% \begin{itemize}

% \item \textit{Task-pilot relationship (temporal)}: What is the lifetime of a
% pilot in relation to the length of a task or workload? \msnote{What kind
% of answers to you expect here in section 4.3?}

% \item textit{Task-pilot relationship (spatial)}: Does a pilot execute one or
% more tasks concurrently?

% \item textit{Supported ``physical'' task types}: What types of (physical) tasks
% are supported, e.g., single-core, MPI, ...?

% \end{itemize}


% \paragraph{Qualitative Properties}

% As qualitative properties we define the properties that are not specifically
% related to \pilot systems but rather describe general concepts commonly found
% in distributed applications, frameworks and tools. The classifiers we have
% picked are:

% \begin{itemize}

% \item \textit{Fault-Tolerance}: What mechanisms are in place to shield the
% system from component failures?

% \item \textit{Security}: How is a user identified in the system and how are
% identities and access credentials delegated to individual resources / pilots?

% \item \textit{Transfer Protocols}: What network protocols that are used to
% connect the individual component of the \pilot system?

% \end{itemize}


\begin{table*}[t]
 \up
 \centering
 \begin{tabular}{|p{3cm}|p{3.2cm}|p{3cm}|p{3cm}|p{3cm}|}
  \hline
  \textbf{Pilot-Job System} &\textbf{Pilot (Resource) Provisioning}    &\textbf{Job-to-Resource (Pilot) Binding}  &\textbf{Single-/Multi-User Pilots} &\textbf{Classifier IV}  \\
  \hline
          AppLeS            & N/A                                      & Automatic (algorithmic/profiling)  & Single (?) & --                     \\
  \hline
          MyCluster         & Manual, Explicitly User-controllable     & Manual                             & Single (?)                  & --                     \\
  \hline
          PanDA             & Automatic, Not user-controllable         & Manual(?)                          & Single (?)  \aznote{\url{http://indico.cern.ch/getFile.py/access?contribId=2&sessionId=6&resId=0&materialId=slides&confId=45473} says otherwise, but I don't think so...}    & --                     \\
  \hline
          GlideinWMS        & Automatic, Not user-controllable         & Manual (rule-driven) / Automatic   & Multi / Single (via Corral) & --                     \\
  \hline
          SWIFT/Coaster     & --                                       & --                                 & --                          & --                     \\
 \hline
          BoSCO             & Manual, Explicitly User-controllable     & Manual (rule-driven) / Automatic   & Multi / Single              & --                     \\
 \hline
          BigJob             & Manual, Explicitly User-controllable     & Manual (rule-driven) / Automatic   & Single              & --                     \\
 \hline
 \end{tabular}
 \caption{\textbf{Different pilot-job systems and their key.
 \onote{Possible classifiers:
 user/system space: can pilots be controlled from system / user space. TODO:
 discuss why we chose specific classifiers!}}
 \up}
 \label{table:implementations-comparison}
\end{table*}

\subsection{Analysis of \pilotjob Implementations}

\mtnote{Introduction to the subsection.}

% -----------------------------------------------------------------------------
%
\subsubsection{\apples}
\begin{itemize}
  \item
    Pilot Characteristics - Lacks true Pilots 
  \item
    Resource Dependences - Application-level resource selection
  \item
    Workload Semantics - Application-level schedule generation 
  \item
    Task Binding Characteristics - Application-level binding 
  \item
    Deployment Strategies - App 
  \item
    Architecture - App 
  \item
    Interfaces - App 
  \item
    Interoperability - App 
  \item
    Multitenancy - App 
  \item
    Robustness - App 
  \item
    Security - App 
  \item
    Usage Modes - Grid 
\end{itemize}
\paragraph{Rationale}
\apples is an ideal introductory software package to be studied
as it is a \textit{pre-\pilot} framework; it is capable of
addressing some of the issues \pilotjobs are designed to handle,
but lacks some fundamental components which limit its usability
compared to the full \pilotjob approach.

\paragraph{Design Goals}
The main design goal of \apples is adaptivity, granting applications 
the ability to generate schedules dynamically and incorporating
application performance characteristics into these schedules.
The concept requires several further
design goals so that changing conditions on heterogeneous
resources can be adapted to, including granting applications
the ability to generate schedules dynamically and incorporating
application performance characteristics into these schedules.

\paragraph{Applications}
\apples has been used to choose destinations
for storing satellite image files (Simple SARA), 
optimize the execution of data-parallel matrix
calculations (Jacobi 2D \apples), 
distribute computational biology genetic searches
across resources with varying response/availability times 
(\apples for Complib), and
analyze biochemical interactions while dynamically recreating
schedules to account for changes in performance predictions
(\apples for MCell).

\paragraph{Deployment Scenarios (VO/multiuser)}
The deployment characteristics of \apples-enabled software rely
mainly upon the application writers themselves; applications
may be adapted to use \apples which are either single or multi-user.
\apples templates exist, which are application-class-based (e.g.
parameter sweep, master-worker) 
software frameworks created
to ease the adaptation of existing applications to make use
of \apples.

\paragraph{Resource Landscape (e.g. grid/cloud/hpc/etc)}
Heterogeneous grid computing resources were targeted by
\apples, with the dynamism inherent in grid resources (availability,
computational power, etc) being a driving 
motivator for \apples itself.  \apples
was first developed in 1996, precluding the development of clouds;
regardless, as with ``true'' \pilotjob approaches, adaptations
and additions which enable cloud functionality could
feasibly be built on top of the \apples framework as e.g.
cloud middleware service interfaces.
 To expand the reach of \apples, middleware-specific
extensions must be implemented.

\paragraph{Architecture \& Interface}
\apples requires a tight coupling between the target application
and the \apples code.  \textit{\apples templates} were created
to ease this process, but the application code in essence must
be modified to make use of \apples, as opposed to \pilotjob
systems which can generally be taken advantage of without
requiring code modification of the underlying applications being
executed.

The lack of a \textit{Pilot} limits the agility
of application-level scheduling, as tasks
must wait in unpredictable resource manager queues before being executed,
undermining the level of control.  For this reason, \apples
is not a ``true'' \pilotjob system, as a \textit{Pilot}
is required to enable fundamental \pilotjob functionality such as
\textit{multi-level scheduling}.

\paragraph{\apples Conclusion}
\apples shows the power of scheduling tasks across multiple computing resources
via the \textit{Master-Worker} approach which many
\pilotjobs use as a partial foundation.  However,
the following ``true'' \pilotjobs will show the flexibility/power
that a full \pilotjob approach offers.

%% \subsubsection{AppLeS}

%% \textcolor{green}
%% {
%% \textbf{Structure}
%% \begin{itemize}
%% \item Functional description: core components -- X does that, Y does that, Z doest that.
%% \item What applications were they designed for?
%% \end{itemize}
%% }

%% \textcolor{red}
%% {
%% \textbf{Usage of 'relevant' terms from section three}
%% \begin{itemize}
%% \item Term 1 =
%% \item Term 2 =
%% \end{itemize}
%% }

%% \textcolor{blue}
%% {
%% \textbf{Mapping to table 1:}
%% \begin{itemize}
%% \item Pilot (Resource) Provisioning:
%% \item Job-to-Resource (Pilot) Binding:
%% \item Singe-/Multi-User pilots:
%% \end{itemize}
%% }

%% AppLeS is a \textit{pre-pilot} framework which allows software built using it to take advantage
%% of enhanced scheduling and resource selection capabilities.
%% With AppLeS, application-level scheduling and resource selection are handled by a
%% per-application \texttt{AppleS agent}.
%% %\aznote{AppLeS agent covers resource discovery, resource selection,
%% %  schedule generation, schedule selection, application execution,
%% %  and schedule adaptation}
%% An \texttt{AppLeS agent} roughly handles the \pilotjob functionality contained in
%% \textit{\pilotjob agents} and \textit{managers}.  This includes resource
%% selection, scheduling tasks to resources, and managing application execution.
%% Job-to-Resource \textit{binding} is handled automatically via the AppLeS agent.
%% %.  In addition, an incomplete subset of
%% %\pilotjob functionality such as application-scheduling is present in AppLeS.
%% AppleS also contains features not fundamental to the \pilotjob paradigm, such
%% as resource discovery via Grid resource discovery services and peformance
%% modelling for resource selection.  The entire package thereby reduces the
%% complexity of assigning tasks to resources from the user's point of view
%% and enables applications to run in a more adaptive fashion.

%% The lack of a \textit{placeholder job} limits the functionality
%% of application-level scheduling, as tasks
%% must wait in unpredictable resource manager queues before being executed,
%% undermining the level of control.  For this reason, AppLeS
%% is not a ``true'' \pilotjob system, as a \textit{placeholder}
%% is required to enable fundamental \pilotjob functionality such as
%% \textit{multi-level scheduling.}
%% However, AppLeS has still been used to choose destinations
%% for satellite image files, optimize the execution of data-parallel matrix
%% calculations, distribute computational biology genetic searches
%% across resources with varying response/availability times, and
%% analyze biochemical interactions while dynamically recreating
%% schedules to account for changes in performance predictions.
%% \aznote{Previous sentence to satisfy request for \pilotjob
%% domain usage}
%% \aznote{Should I put something here like
%% ``AppLeS has been successful in these domains due in part
%% to the usefulness of the M-W paradigm which \pilotjobs
%% are based on, but the following ``True'' \pilotjobs
%% illustrate the extended reach/power made available
%% through a proper \pilotjob approach''?}

%% %% \aznote{Lack of multi-level scheduling asserted because even
%% %% though AppLeS allows for application-level scheduling,
%% %% it is still submitted to usual resource manager queues, not an internal
%% %% ``pilot'' queue}
%% %% Such enhancements
%% %% are made in the ``true'' \pilotjob systems which succeed it,
%% %% offering additional adaptibility and user-control over that
%% %% made possible by AppLeS.
%% \aznote{Early vs. late binding here -- I am not sure.  AppLeS generates
%% a task-resource mapping early before jobs are even running + then sticks
%% to it making it ``early binding'' in my mind,
%% even though it does so after profiling the application.  However, according
%% to the vocab, if the manager/agents choose which resources will be used
%% it's ``late binding''.  So that is what I will go with.}

%% %\aznote{\url{http://www.sc2000.org/techpapr/papers/pap.pap169.pdf} - Also consider this paper:
%% %The AppLeS Parameter Sweep Template: User-Level Middleware for the Grid?}

%% % -----------------------------------------------------------------------------
%% %

\subsubsection{MyCluster}
\begin{itemize}
  \item
    Pilot Characteristics - N/A 
  \item
    Resource Dependences - Condor/SGE resources
  \item
    Workload Semantics - ``Virtual Login'', CLI user-driven task submission 
  \item
    Task Binding Characteristics - Late-binding, pilots must be up before submitting 
  \item
    Deployment Strategies - Task manager/slave-node manager  
  \item
    Architecture - Client-server, agent manager server proxy manager clients, etc 
  \item
    Interfaces - CLI-user-driven 
  \item
    Interoperability - Condor/SGE, uni-pilot 
  \item
    Multitenancy - Single-user 
  \item
    Robustness - Each proxy at gateway maintains pilots in cases of network outage, gateway proxies
    may go up/down in isolation 
  \item
    Security - GSI-Authenticated TCP, 64-bit key checking 
  \item
    Usage Modes - HPC/Grid 
\end{itemize}

\paragraph{Rationale}
MyCluster was developed to allow users to submit and manage
jobs across heterogeneous NSF TeraGrid resources in a uniform,
on-demand manner.  TeraGrid existed as a group of compute clusters
connected by high-bandwidth links facilitating the movement
of data, but with many different job deployment middlewares
requiring cluster-specific submission scripts.  MyCluster
allowed all cluster resources to be submitted to via a
\pilotjob-based approach, reducing the complexity of
multi-cluster submission and increasing scheduling flexibility.
%  in this case,
%the resources are spread across a national DCI.

\paragraph{Design Goals}
The authors describe their system
as a ``personal cluster'' with the goal of achieving on-demand,
user-schedulable computing resources aligning closely with what is
provided by our formal definition of \pilotjobs.  This enhancement
to user control was envisioned as a means of allowing
users to submit and manage thousands of jobs at once
across heterogeneous distributed computing infrastructures, while
providing an familiar interface for submission.

\paragraph{Applications}
Applications are launched via MyCluster in a ``traditional''
HPC manner, via a \texttt{virtual
login session} which contains usual queuing commands to submit
and manage jobs.  This means that applications do not need
to be explicitly rewritten to make use of MyCluster functionality;
rather, MyCluster provides user-level \pilotjob capabilities
which users can then use to schedule their applications with.
%% \aznote{There appear to only be synthetic benchmarks
%% in cited paper; spent some time going through
%% papers which reference MyCluster and did not find
%% any specific applications/papers which explicitly used
%% MyCluster.  Is this a problem?}

\paragraph{Deployment Scenarios (VO/multiuser)}
MyCluster is designed for a single-user.  A MyCluster
installation resides in userspace, and may be used
to marshal multiple resources ranging from small local resource pools
(e.g. departmental Condor or SGE systems) to large HPC installations
including TeraGrid (now XSEDE).

\paragraph{Resource Landscape (e.g. grid/cloud/hpc/etc)}
MyCluster was designed for and successfully executed on NSF TeraGrid
resources, enabling large-scale cross-site submission of ensemble
job submissions via its virtualized cluster interface.
This approach makes the \textit{multi-level scheduling} abilities
of \pilotjobs explicit; rather than directly \textit{binding}
to individual TeraGrid resources, users allow the virtual grid
overlay to \textit{schedule} tasks to multiple allocated TeraGrid sites
presented as a single cohesive, unified resource.

\paragraph{Architecture \& Interface}
MyCluster implements several \pilotjob analogues: the \texttt{Master
  Node Manager}, which acts as a \textit{\pilotjob Manager}, the
\texttt{Task Manager}, which acts as an \textit{\pilotjob Agent},
\texttt{Job Proxies}, which act as \textit{placeholders}, and
\texttt{Slave Node Managers}, which handle \textit{\pilotjob Tasks}
Both resource provisioning and job-to-resource \textit{bindings}
are handled manually by the end user.
The end result is a complete \textit{basic \pilotjob system}, despite the
authors of the system being constructed not having used the word ``pilot''
once!~\cite{1652061}

Additional features above and beyond the \textit{basic \pilotjob system}
concept include: the inclusion of a MyCluster
\texttt{virtual login session}, which
is a commandline interface designed to allow users to interactively
monitor the status of their \textit{placeholders} and \textit{tasks},
the automatic recovery of placeholders after loss due to exceeding wallclock
limitations or node reboots; and authentication of all TCP
connections used in the system via 64-bit key encryption and
GSI authentication mechanisms.

\paragraph{MyCluster Conclusion}
MyCluster illustrates how an approach aimed at \textit{multi-level
scheduling} by marshalling multiple heterogeneous resources lends
itself perfectly to a \pilotjob-based approach.  The fact that
the researchers behind it formed a complete \pilotjob system
while working toward interoperability/uniform access is a testament
to the usefulness of \pilotjobs in attacking these problems.

%% \subsubsection{MyCluster}

%% \textcolor{red}
%% {
%% \textbf{Mapping to section 3:}
%% \begin{itemize}
%% \item Term 1 =
%% \item Term 2 =
%% \end{itemize}
%% }

%% \textcolor{blue}
%% {
%% \textbf{Mapping to table 1:}
%% \begin{itemize}
%% \item Pilot (Resource) Provisioning:
%% \item Job-to-Resource (Pilot) Binding:
%% \item Singe-/Multi-User pilots:
%% \end{itemize}
%% }

%% %% \aznote{NOT EVEN SURE THIS IS A ``BASIC PILOT-JOB SYSTEM'' right now:
%% %% ``Our system delegates the responsibility of submitting the
%% %% job proxies to a single semi-autonomous agent spawned,
%% %% through Globus, at each Te raGrid site during a virtual login
%% %% session. This agent translates the job proxy submission to the
%% %% local batch submission syntax, maintains some number of job
%% %% proxies throughout the lifetime of a login session, and may
%% %% negotiate migration of job proxies between sites based on
%% %% prevailing job load conditions across sites.''
%% %% Rather, it looks like MyCluster provides some method of
%% %% submitting jobs to resources in a unified manner, much like
%% %% AppLeS.  When it comes time for the job to run, however,
%% %% it must wait in the local queuing system...?  In short,
%% %% looks like abstraction across multiple resources with some
%% %% functionality to handle everything from a single unified
%% %% access point including data handling, but
%% %% there is no concept of a placeholder/agent.
%% %% }

%% %% \aznote{I still agree with my above statements after additional reading --
%% %% it looks like this is kind of a ``virtualization'' system for grids
%% %% where you can submit/monitor jobs via a centralized interface and lacking
%% %% in what we consider vital to be a full-fledged Pilot-Job system.  Would
%% %% welcome discussion here.}

%% MyCluster is a \textit{basic \pilotjob system} which is a direct
%% abstraction of cluster-based distributed computing systems.
%% %% The stated goal of the project is to
%% %% ``build personal [...] clusters on demand''
%% %% \aznote{from webpage} and in doing so
%% %% inadvertantly implements \pilotjob functionality to accomplish
%% %% its aims.
%% The authors describe their system
%% as a ``personal cluster'' with the goal of achieving on-demand,
%% user-schedulable computational resources aligning closely with what is
%% provided by our more formal definition of \pilotjobs.

%% MyCluster implements several \pilotjob analogues: the \texttt{Master
%%   Node Manager}, which acts as a \textit{\pilotjob Manager}, the
%% \texttt{Task Manager}, which acts as an \textit{\pilotjob Agent},
%% \texttt{Job Proxies}, which act as \textit{placeholders}, and
%% \texttt{Slave Node Managers}, which handle (are?) \textit{\pilotjob
%%   \cus}.  Both resource provisioning and job-to-resource \textit{bindings}
%% are handled manually by the end user.
%% The end result is a complete \textit{basic \pilotjob system}, despite the
%% authors of the system being constructed not having used the word ``pilot''
%% once!~\cite{1652061}

%% Additional features above and beyond the \textit{basic \pilotjob system}
%% concept include: the inclusion of a MyCluster
%% \texttt{virtual login session}, which
%% is commandline interface designed to allow users to interactively
%% monitor the status of their \textit{placeholders} and \textit{\cus};
%% the automatic recovery of placeholders after loss due to exceeding wallclock
%% limitations or node reboots; and authentication of all TCP
%% connections used in the system via 64-bit key encryption and
%% GSI authentication mechanisms.

%% MyCluster was designed for and successfully executed on NSF TeraGrid
%% resources, enabling large-scale cross-site submission of ensemble
%% job submissions via its virtualized cluster interface.
%% This approach makes the \textit{multi-level scheduling} abilities
%% of \pilotjobs explicit; rather than directly \textit{binding}
%% to individual TeraGrid resources, users allow the virtual grid
%% overlay to \textit{schedule} tasks to multiple allocated TeraGrid sites
%% as a whole.

%% %\textit{Manager} functionality is handled by MyCluster's
%% %\texttt{Master Node Manager}.

%% %% Shortcomings include ...  (lack of heterogenous support?)

%% %% \begin{itemize}
%% %% \item Master Node Manager = Manager \aznote{Master Node
%% %% Manager process within a GNU Screen session that then
%% %% starts the Condor/SGE master daemons in user space if
%% %% needed.}
%% %% \item Agent Manager \aznote{ The
%% %% agent manager remotely executes a Proxy Manager process at
%% %% each participating site using Globus GRAM}
%% %% \item Submission Agent -- \aznote{Wrapper for GridShell, wraps
%% %%   multiple grid resources to a single ``virtual'' cluster
%% %% \url{https://www.opensciencegrid.org/bin/view/Interoperability/GridShellEvaluation}}
%% %% \item Proxy Manager = one at each site, reports back to central
%% %%   agent manager w/ heartbeat to keep awareness of resource state
%% %% \item Task Manager = \textit{agent}
%% %% \item Slave-node managers = compute units (?) (probably not,
%% %%   barely mentioned in the paper)
%% %%   \aznote{The slave node managers are responsible for starting the
%% %%     Condor or SGE job-starter daemons. These daemons connect
%% %%     back to the master processes on the client workstation, or to a
%% %%     pre-existing departmental cluster, depending on the virtual
%% %%     login configuration.
%% %%   }
%% %% \item Job proxy (?)
%% %% \end{itemize}

%% % -----------------------------------------------------------------------------
%% %

\subsubsection{\panda}

\begin{itemize}
  \item
    Pilot Characteristics - ? 
  \item
    Resource Dependences - Condor Glide-in, supports PBS/LSF/others through this, plugin support
  \item
    Workload Semantics - job type, priority, input data/locality,  
  \item
    Task Binding Characteristics - Late-binding (?)  
%When the pilot is launched, it collects information about the worker node and sends it to
%the job dispatcher. If a matching job does not exist, the pilot will end. If a job does exist, the pilot will
%fork a separate thread for the job and start monitoring its execution.
  \item
    Deployment Strategies - DDM handles data transfer, must have database access,
    file stage-in/out + cleanup 
  \item
    Architecture - Client-server 
  \item
    Interfaces - Python/HTTP Client 
  \item
    Interoperability - Condor/SGE, uni-pilot 
  \item
    Multitenancy - Multi-user 
  \item
    Robustness - Disk space monitoring, job recovery of failed jobs from restarting
    from crash if possible 
  \item
    Security - GSI Certificate-based security, HTTPS, glExec to run as given user 
  \item
    Usage Modes - HTC, Cloud, HPC WIP(?) 
\end{itemize}

\paragraph{Rationale}
\panda was developed to provide a multi-user
\pilotjob system for ATLAS~\cite{aad2008atlas}, which is
a particle detector at the Large Hadron Collider
which could
handle large numbers
of jobs for data-driven processing workloads.
In addition to the logistics of handling large-scale job submission, there was
a need for integrated monitoring for analysis of system state and a high
degree of automation to reduce the need for user/administrative intervention.

\paragraph{Design Goals}
\panda was designed as an advanced \pilotjob system,
satisfying requirements such as the ability
to manage data, monitor job/disk space, and recover from failures.
A modular approach allowing the use of plug-ins was chosen in order
to incorporate additional features in the future.
\panda was designed from the ground-up to meet these requirements
while scaling to handle the large scale of jobs and data produced
by the ATLAS experiment.

\paragraph{Applications}
\panda has been used to process data and jobs relating to ATLAS.  Approximately
a million jobs a day are managed~\cite{pandapresentation2013-06} 
which handle simulation, analysis, and other work~\cite{maeno_pd2p:_2012}. 
The ATLAS experiment itself produces several petabytes of data a year
which must be processed and analyzed.

\paragraph{Deployment Scenarios (VO/multiuser)}
\panda has been initially deployed as an HTC-oriented
multi-user
WMS system for ATLAS, consisting
of ~100 heterogeneous computing sites~\cite{maeno_pd2p:_2012}.
Recent improvements to \panda have been designed to extend 
the range of deployment scenarios to non-ATLAS infrastructures
making \panda a general-use \pilotjob~\cite{nilsson2012recent}.
%% \aznote{Per-user/per-working group reference made in slides, follow-up
%% with actual papers}

\paragraph{Resource Landscape (e.g. grid/cloud/hpc/etc)}
\panda began as a specialized \pilotjob for the ATLAS grid, and has been extended
into a generalized \pilotjob which is capable of working across other grids
as well as HPC and cloud resources.
Cloud capabilities have been used as part
of Helix Nebula (CloudSigma, T-Systems, ATOS),
the FutureGrid and Synnefo clouds, and the commercial
Google and EC2 cloud offerings as well~\cite{pandapresentation2013-06},
extending the reach of \panda to
cloud resources as well.  Future \panda developments seek
to interoperate with HPC resources~\cite{pandapresentation2013-06}.

\paragraph{Architecture \& Interface}
\panda's \textit{manager} is called a \texttt{\panda server}, and matches
jobs with \pilots in addition to handling data management.
\panda's \textit{\pilot}
is called, appropriately enough, a \texttt{pilot}, and handles the execution
environment.  These pilots are generated via \panda's \texttt{PilotFactory},
which also monitors the status of pilots.
Pilot-resource \textit{provisioning} is handled by \panda itself and is not
user-controllable, whereas job-to-resource \textit{binding} is handled
manually by users.
A central job queue allows users to submit jobs
to distributed resources in a uniform manner. 
%%  \aznote{Not required for
%% \pilotjob -- put later or remove?}
This basic functionality (pilot creation and management) provides
\panda with all of the baseline capabilities required for a \pilotjob.

As \panda is an \textit{Advanced \pilotjob System}, it enables functionality
beyond that of a \textit{Basic \pilotjob}.
\panda contains some additional backend features such as \texttt{AutoPilot}, which
tracks site statuses via a database, \texttt{Bamboo} which adds
ATLAS database interfacing, and automatic error handling and recovery.
\panda contains support for queues in clouds, including EC2, Helix Nebula,
and FutureGrid~\cite{pandapresentation2013-06}.
%% \aznote{Have to find a good cite for this}
Enhancements to the userspace include
\texttt{Monitor}, for web-based monitoring, and the \texttt{\panda client}.
\texttt{\panda Dynamic Data Placement} \cite{maeno_pd2p:_2012}
allows for additional, automatic data
management by replicating popular or backlogged input data to underutilized resources
for later computations and enables jobs to be placed where the data
already exists.


\paragraph{\panda Conclusion}
\panda has extended beyond its origins at the ATLAS experiment, and has also been used for other projects such
as the Open Science Grid.  This illustrates the importance of designing
\pilotjobs which are easily adaptable such that \pilots are not tied
to any particular DCI; in this manner, their advantages can be made 
available beyond their original design scope.

\textcolor{red}
{
\textbf{Mapping to section 3:}
\begin{itemize}
\item Term 1 =
\item Term 2 =
\end{itemize}
}

\textcolor{blue}
{
\textbf{Mapping to table 1:}
\begin{itemize}
\item Pilot (Resource) Provisioning:
\item Job-to-Resource (Pilot) Binding:
\item Singe-/Multi-User pilots:
\end{itemize}
}

%\aznote{Discussion on PanDA's design for LHC data processing and why \pilotjobs
%work for this, or is this out-of scope?}


%% \textbf{job, task, resource, infrastructure, scheduling, pilot, pool, manager,
%%   agent, Pilot-Job, placeholder, multi-level scheduling, binding, early/late
%%   binding}

%% \begin{itemize}
%% \item Based upon Condor-G/Glidein, uses multiple queues
%% \item Jobs submitted via Condor-G
%% \item Jobs run with wrappers, on worker nodes they download/execute PanDA pilot code
%% \item Pilot then asks for job that worker node can handle
%% \item Stages in input files; executes and monitors progress of job; stages out output;
%%   cleans up environment
%% \item ``Special Features''
%%   \begin{itemize}
%%   \item Job/disk monitoring
%%   \item Job recovery (leaves output files resident for another
%%     process to resume from)
%%   \item Multi-job processing (submit batch of jobs to a single pilot)
%%   \item Certificate/token-based security
%%   \item gLExec security (user's identity used instead of Pilot credentials)
%%   \end{itemize}

%% \item PanDA server (task buffer -- job queue manager keeps track of
%%   active jobs, brokerage -- matches jobs  with sites/pilots and manages
%%   data placement, job dispatcher -- receives requests from pilots and sends
%%   jobs payloads, data service -- dispatch and retrieval from sites)
%% \item PanDA DB (general DB = advert?)
%% \item PanDA client (user client for submission/etc)
%% \item Pilot (execution environment)
%% \item AutoPilot (pilot submission, management, monitoring), submits
%%   pilots to remote sites and tracks site status via database
%% \item SchedConfig (configuration database)
%% \item Monitor (Web-based monitoring)
%% \item Logger (logs incidents w/ Python logging)
%% \item Bamboo (interface w/ ATLAS database)
%% \item PilotFactory -- generates pilots (sends schedd glideins to remote sites)
%%   and monitors pilot status, basically enables use of glideins with PanDA (?)
%% \end{itemize}



% -----------------------------------------------------------------------------
%
\subsubsection{HTCondor}

HTCondor can be considered one of the most prevalent distributed computing
project of all time in terms of its pervasiveness and size of its  user
community. Is often cited as the project that coined the term \pilotjob. But
despite this fact, describing HTCondor along the lines of a \pilotjob system has
turned out to be difficult, if not partly impossible for several reasons.

HTCondor is used in multiple different contexts: the HTCondor \textit{project},
the HTCondor \textit{software} and HTCondor \textit{grids}. But even if we only
look at the software parts of the landscape, we are faced with a
\textit{plethora} of concepts, components and services that have been grown and
curated opportunistically for the past 20 years.

% -----------------------------------------------------------------------------
%
\paragraph{Core System}

A ``complete'' HTCondor system that accepts user tasks and executes them on
one or more resources is called an HTCondor \textbf{pool}. A pool consists of
multiple loosely coupled, usually distributed  components that together
implement the functionality of Workload Manager and Task Executor in a
PilotJob system.

In a pool, user tasks (\textbf{jobs}) are represented through job description
files and submitted to a (user-)\textbf{agent} via command-line tools. Agents
are deployed as system services (\texttt{schedd}) on so-called \textit{gateway
machines} (user front-ends of an HTCondor pool) and accept tasks from multiple
users. Agents implement three different aspects of the overall architecture:
(1) they provide persistent storage for user jobs, (2) they find resource for
a task to execute by contacting the \textbf{matchmaker} and (3) they marshal
task execution via a so- called \textbf{shadow}.

The matchmaker, also called the \textit{central manager}, is another system
service that realizes the concept of late binding by matching user tasks with
one or more of the resources available to an HTCondor pool. The matchmaking
process is based on \textit{ClassAds} a description language that can capture
both, resource capabilities as well as task requirements.

\textbf{Resources} are tied into an HTCondor pool by \texttt{startd} system
services that are analogous to our definition of a \pilot. The \texttt{startd}
\pilots are deployed on the pool's compute resources. They report resource
state and  capabilities back to the matchmaker and start tasks submitted by
agents on the resource in encapsulated \textbf{sandboxes}.

Resources in a pool can span a wide spectrum of system. While some pools are
comprised of regular desktop PCs (sometimes called a campus grid), other pools
incorporate large HPC clusters and cloud resources. Hybrid pools with
heterogeneous sets of resources are also common.

It is possible for two or more HTCondor pools to ``collaborate'' so that one
pool has access to the resources of another pool and vice versa. In HTCondor,
this concept is called \textbf{flocking} and allows agents to query
matchmakers outside their own pool for compatible resources. Flocking is used
to implement load-balancing between multiple pools but also to provide a
broader set of heterogeneous resources to user communities.

The components described above, jobs, agent, matchmaker, resource, shadow and
sandboxes are sometimes collectively referred to as the \textbf{HTCondor
Kernel} and satisfy the requirements for a \pilotjob system as outlined in
Section \ref{subsec:vocab_core_functionalities}. However, the provisioning,
allocation and usage of resources within a pool can differentiate between
different pools and multiple different approaches and software systems have
emerged over time, all under the umbrella of the wider HTCondor project.

% -----------------------------------------------------------------------------
%
\paragraph{Condor-G -- Condor via Globus}

Condor-G is an alternative (user-)agent for HTCondor that can ``speak'' the
Globus GRAM (Grid Resource Access and Management) protocol. GRAM services are
often deployed as remote job submission endpoints on top of HPC cluster
queuing systems. Condor-G allows users to incorportate those HPC resources
temporarily to an HTCondor pool.

Condor-G agents use the GRAM protocol to launch HTCondor \texttt{startd}
\pilots ad hoc via a GRAM endpoint service on a remote system. Tasks submitted
through the Condor-G (user-)agent are then assigned by a local matchmaker to
these ad-hoc provisioned \pilots. This concept is called \textit {gliding-in}
or \textbf{glide-in}. It implements \textbf{late-binding} on top of GRAM/HPC
systems: the (user-)agent can assign tasks to \texttt{startd}s \textit{after}
they have been scheduled and started through the HPC queueing system. This
effectively decouples resource allocation (\texttt{startd} scheduling through
GRAM) and task assignment.

% -----------------------------------------------------------------------------
%
\paragraph{glidein-WMS -- Automated \pilot Provisioning}

The glidein workload management system (WMS)~\cite{1742-6596-119-6-062044} introduces advanced \pilotjob capabilities to HTCondor by 
providing automated \pilot (\texttt{startd}) provisioning based on
the state of an HTCondor pool.


% -----------------------------------------------------------------------------
%
\paragraph{BoSCO}

BoSCO is a user-space job submission system based on HTCondor. BoSCO was
designed to allow individual users to utilize heterogeneous HPC and grid
computing resources through a uniform interface. Supported backends include
PBS, LSF and GridEngine clusters as well as other  grid resource pools managed
by HTCondor. BoSCO supports both, an agent-based (\textit{glidein} / worker)
and a native job execution mode through a single user-interface.

BoSCO exposes the same \textit{ClassAd}-based user-interface as HTCondor,
however, the backend implementation for job management and resource
provisioning is significantly more lightweight than in HTCondor and explicitly
allows for ad hoc user-space deployment. BoSCO provides a \pilotjob-based
system that does not require the user to have access to a centrally-
administered HTCondor campus grid or  resource pool. The user has direct
control over \pilotjob agent provisioning (via the \texttt{bosco\_cluster}
command) and job-to-resource binding via \textit{ClassAd} requirements.

The overall architecture of BoSCO is very similar to that of HTCondor. The
\textit{BoSCO submit-node} (analogous to Condor \texttt{schedd}) provides the
central job submission service and manages the job queue as well as the worker
agent pool. Worker agents communicate with the \textit{BoSCO submit-node} via
pull-requests (TCP). They can be dynamically added and removed to a
\textit{BoSCO submit-node} by the user. BoSCO can be installed in user-space
as well as in system space. In the former case, worker agents are exclusively
available to a single user, while in the latter case, worker agents can be
shared among multiple users. The client-side tools to submit, control and
monitor BoSCO jobs are the same as in Condor (\texttt{condor\_submit},
\texttt{condor\_q}, etc).


%\textit{Collectors} are the managers, with one collector per
%GlideinWMS system.
%\textit{Glidein factory daemon}s are responsible for
%submitting pilot jobs to grid pools\aznote{Use diff vocab for grid pool?}.
%They therefore correspond to pilot managers.
%\textit{VO frontend daemons} are the schedulers, which map jobs
%to pilots.  GlideinWMS takes advantage of late-binding, as
%jobs are not selected for resources until the pilots start.\aznote{This
%should be true of all systems described in this section -- do we
%need to explicitly say so for each pilot system?}
%\textit{WMS collector machine} manages communication between the glidein
%factory daemons(pilot managers) and VO frontend daemons(schedulers)
%and so corresponds to \aznote{We don't have anything for this.  This
%sounds like what REDIS/etc do for BJ though.  Should we incorporate this
%into our vocabulary?}
%\textit{User jobs} correspond to compute units \aznote{Right?}

\begin{itemize}
%Unlisted things \\
\item Workflow (doesn't seem to be explicit support)
\item Workload (is this just a list of user jobs that are queued up)
\item Placeholder (isn't this the same kind of deal as a Pilot-Job/PJ
waiting in a queue?)
\item Pilot framework (The entire GlideinWMS qualifies as a ``pilot
framework'', correct?)
\item Ensemble \aznote{How to tie this in}
\item Platform \aznote{Ditto}
\item Master-Worker \aznote{Likewise, we could say that the
VO factory daemons are the masters with the pilots the workers,
but this seems overly pedantic?}
\end{itemize}

\textbf{CorralWMS:} CorralWMS is an alternative front-end for GlideinWMS-based
infrastructures. It replaces or complements the regular GlideinWMS front-end
with an alternative API which is targeted towards workflow execution. Corral was
initially designed as a standalone pilot (glidein) provisioning system for
the Pegasus workflow system where user  workflows often produced workloads
consisting of many short-running jobs as well as mixed workloads consisting of
HTC and HPC jobs.

Over time, Corral has been integrated into the GlideinWMS stack as CorralWMS.
While CorralWMS still provides the same user-interface as the initial, stand-
alone version of Corral, the underlying pilot (glidein) provisioning is
now handled by the GlideinWMS factory.

The main differences between the GlideinWMS and the CorralWMS front-ends lie in
identity management and resource sharing. While GlideinWMS pilots (glidins) are
provisioned on a per-VO base and shared / re-used amongst members of that VO,
CorralWMS pilots (glideins) are bound to one specific user via personal  X.509
certificates. This enables explicit resource provisioning in non-VO centric
environments, which includes many of the HPC clusters that are part of U.S.
national cyberinfrastructure (e.g., XSEDE).


% -----------------------------------------------------------------------------
%
%\subsubsection{Corral}
%\onote{Corral seems to be a component in the glidein-WMS landscape and
%not an independent pilot-job implementation. I don't think that we should
%dedicate an extra section to it.}
%\aznote{SJ asked for an investigation of Corral -- not sure that this
%deserves a full analysis at this point on a technical level despite
%being commonly used, open to suggestions}

%Corral is designed to allow hybrid HTC/HPC execution, in which
%many small jobs may be executed in conjunction with larger runs.
%\cite{Rynge:2011:EUG:2116259.2116599}
%\aznote{Back this up w/ paper refs -- paper is somewhat dated,
%verify this is still true today}.  Corral operates as a Glidein
%WMS frontend\aznote{main reason that I think we shouldn't include
%Corral in its own section...}, where GlideinWMS manages the size
%of Condor glide-in pools.
%\begin{itemize}
%\item Workflow - handled by Pegasus workflow management system
%  \aznote{True in the paper I am using, but not handled by Corral itself, so should we include this?}
%\item Placeholder - handled by multislot requests
%  \aznote{multislot request requests a single large
%    GRAM job, and then starts glideins within this container}
%\item Job
%\item Compute unit
%\item Workload
%\item Will fill this in later...
%\end{itemize}


% -----------------------------------------------------------------------------

% AS PER SJ P* CALL: COASTERS SECTION IS CANCELED
% mrnote: Reason - insufficient documentation, no citations to actual science problems
%\subsubsection{Coasters}
%
%\textcolor{red}
%{
%\textbf{Mapping to section 3:}
%\begin{itemize}
%\item Term 1 =
%\item Term 2 =
%\end{itemize}
%}
%
%\textcolor{blue}
%{
%\textbf{Mapping to table 1:}
%\begin{itemize}
%\item Pilot (Resource) Provisioning:
%\item Job-to-Resource (Pilot) Binding:
%\item Singe-/Multi-User pilots:
%\end{itemize}
%}
%
%\msnote{Main coasters ref: \url{http://www.ci.uchicago.edu/swift/papers/UCC-coasters.pdf}}
%\mrnote{Coasters wiki: \url{http://wiki.cogkit.org/wiki/Coasters}}
%\mrnote{WIP: Bullet points that will later be converted to text}
%
%\paragraph{Rationale}
%
%The Coaster System (or "Coasters") is a Java CoG based Pilot-Job system
%created for the needs of the Swift parallel scripting language.
%
%\paragraph{Design Goals}
%
%\begin{itemize}
%  \item Driven by the needs of Swift
%  \item Automatically-deployed to endpoint. Does not need user login to endpoint.
%  \item Supports file staging, on-demand opportunistic multi-node allocation, 
%  remote log gin, and remote monitoring
%\end{itemize}
%
%\paragraph{Applications}
%
%\begin{itemize}
%  \item It has been used since 2009 for applications in fields 
%that include biochemistry, earth systems science, 
%energy modeling, and neuroscience.
%  \item Above claim made in Coasters paper - need to obtain citations
%\end{itemize}
%
%\paragraph{Deployment Scenarios}
%
%
%\paragraph{Resource Landscape (e.g. grid/cloud/hpc/etc)}
%
%\begin{itemize}
%  \item grids, clouds, and ad-hoc desktop-computer networks 
%  \item OSG, TeraGrid/XSEDE, Blue Gene/P, Cray XT and XE, 
%  Sun Constellation, small clusters, clouds (BioNimbus, FutureGrid,
%  Amazon EC2)
%\end{itemize}
%
%\paragraph{Architecture \& Interface}
%
%\textit{Coaster Service} is deployed on remote clusters.
%
%\textit{Coaster Client} is deployed locally.
%\textit{Coaster Workers}
%
%
%\paragraph{Coasters Conclusion}
%\mrnote{WIP}


\subsubsection{Falkon}

\mrnote{Main Falkon ref: \url{http://dev.globus.org/images/7/78/Falkon_SC07_v42.pdf}}
%\mrnote{Best Falkon ref: \url{http://dev.globus.org/wiki/Incubator/Falkon}}

\paragraph{Rationale}
The Fast and Light-weight tasK executiON framework (Falkon)
~\cite{1362680} was created with the primary objective
of enabling many independent tasks to run on large computer
clusters (an objective shared by most \pilotjob systems). 
A particular focus on performance and time-to-completion for jobs
on such clusters drove Falkon development. In 
addition to being \textit{fast}, Falkon, as its name suggests, 
also focused on lightweight deployment schemes.

\paragraph{Design Goals}
As previously stated, the design of Falkon was centered
around the goal of providing support to run large numbers
of jobs efficiently on large clusters and grids. Falkon
realizes this goal through the use of (i) a 
dispatcher to reduce the time to actually place
tasks as jobs onto specific resources (such a feature
was built to account for different issues amongst
distributed cyberinfrastructure - such as multiple
queues, different task priorities, allocations, etc), 
(ii) a provisioner which is responsible for
resource management, and (iii) data caching in
a remote environment~\cite{1362680}.


\paragraph{Applications}

Falkon has been shown to work with many large-scale
applications across various domains. Falkon 
has been integrated with the Karajan workflow language
and execution engine, meaning that applications
that utilize Karajan to describe their workflow
will be able to be executed by Falkon. Simpler
task execution can be achieved without 
modifying the existing executables - sufficient
task description in the web service is all that 
is required to utilize the Falkon system. 

Falkon has been tested for throughput 
and performance in such as applications
as fMRI (medical imaging), Montage (astronomy workflows),
and MolDyn (molecular dynamics simulation) and has shown
favorable results in terms of overall execution time when compared
to GRAM and GRAM/Clustering methods~\cite{1362680}.

\paragraph{Deployment Scenarios}

The Swift parallel programming system~\cite{Wilde2011} was integrated
with Falkon for the purpose of task dispatch. The overall provider
mechanism of Falkon is roughly 840 lines of Java code and meant
to be as lightweight as possible. The Dispatcher service 
in Falkon is implemented by means of a web service. 
This Dispatcher implements a factory/instance deployment scenario.
When a new client sends task submission information, a new instance
of a Dispatcher is created. Each instantiation of the Dispatcher maintains
its own task queue and state - in this way, Falkon can be considered
a single-user deployment scheme, wherein the ``user'' in this case
refers to an individual client request.

\paragraph{Resource Landscape (e.g. grid/cloud/hpc/etc)}
Falkon was originally developed for use on large computer clusters
in a grid environment, but has since been expanded to work on
clouds and other 

Falkon has been 
shown to run on TeraGrid (now XSEDE), TeraPort, Amazon EC2, IBM
Blue Gene/L, SiCortex, and Workspace Service~\cite{1362680}.
\mrnote{GitHub Issue 23} Work has also been done on Falkon
to expand its data capabilities. In addition to data caching and
efficient data scheduling techniques, Falkon has adopted a
data diffusion approach. Using this approach, resources for
both compute and data are acquire dynamically and compute
is scheduled as close as possible to the data it requires.
If necessary, the data diffusion approach replicates data
in response to changing demands~\cite{raicu2008accelerating}.


\paragraph{Architecture \& Interface}

Falkon's architecture relies on the use of multi-level scheduling
as well as efficient dispatching of tasks to heterogeneous DCIs.
As mentioned above, there are two main components of Falkon: 
(i) the Dispatcher for farming out tasks and 
(ii) the Provisioner for acquiring resources. 

The overall task submission mechanism can be considered
a 2-tier architecture; the Dispatcher (using the above terminology,
this is the \pilot-Manager) and the Executor (the \pilot-Agent). 
The Dispatcher is a GRAM4 web service whose primary
function is to take task submission as input and farm
out these tasks to the executors. The Executor runs on
each local resource and is responsible for the actual
task execution. Falkon also utilizes \textit{provisioning}
capabilities with its Provisioner. The Falkon Provisioner is the 
closest analogous entity to a \pilot: it is the creator and 
destroyer of Executors, and is capable of 
providing both static and dynamic resource
acquisition and release.

Falkon has also extended itself beyond basic \pilotjob functionalities
and supports a fault tolerance mechanism which suspends and
dynamically readjusts for host failures. The \textit{data management}
capabilities of Falkon also extend beyond the core \pilotjob
functionalities as described above. Further, in order 
to process more complex workflows, Falkon has been integrated
with the Karajan workflow execution engine~\cite{karajan}. This
integration allows Falkon to accept more complex workflow-based
scientific applications as input to its \pilot-like job execution
mechanism.

\paragraph{Falkon Conclusion}
Falkon contains many of the basic functionalities required to 
qualify it as a \pilotjob system, in addition to some advanced resource
provisioning capabilities, fault tolerance, and workflow-execution integration.
Falkon has been demonstrated to achieve throughput in the range
of hundreds to thousands of tasks per second for very fine-grained
tasks. The per task overhead of Falkon execution has been shown
to be in the millisecond range. Falkon has extended its capabilities
to encompass advanced data-scheduling and caching. Falkon does
not support MPI jobs, however - a limiting factor in its adoption to
certain scientific applications.

 -----------------------------------------------------------------------------

\subsubsection{BigJob}


%\mrnote{Stabbing at the new format in note form}

\begin{itemize}
\item Application: a BigJob script that can also involve logic
and algorithms external to PJ system communication
\item Workload: A set of CUs
\item Task: CU
\item Resource: The resource is whatever is user-specified in the PCD, provided it
follows the `Resource Dependencies' thing
\item Infrastructure: Where BJ runs (grids, clouds, clusters)
\end{itemize}


The core properties of BigJob are as follows:

\begin{itemize}
\item Pilot Characteristics can be user-defined using a
Pilot Compute Description. Define what that is.
Pilot characteristics are often limited by policies on the infrastructures
in which they run - i.e. the size and lifespan of a
pilot may be limited by max number of cores
that can be reserved or max time of a job in local queue

\item Resource dependencies: Needs python >2.5 on remote resource. Supports submission to the following schedulers: SLURM, SGE, PBS/TORQUE, \mrnote{Other schedulers are eluding me at the moment}, also runs in fork:// or ssh:// mode. Submission to EC2, GCE, and Euca clouds. Here is an attempt.
BigJob bootstraps itself into user-space on external targeted 
infrastructures, provided that the python version is greater than 2.5.

\item C\&C: Redis, centralized database approach. Communication from agents (pilots) to
Redis to pull tasks.

\item Workload semantics: User-defined location affinity, wherein labels
can be given to pilots and tasks, so that if the user thinks a CU
should run on a certain pilot, the ComputeDataService scheduler
tries to place these together. 

\item Task Binding Characteristics: Late-binding only.
Tasks cannot be bound to a Pilot before the Pilot switches to active mode on the target
resource.  \mrnote{Ask someone if this is true.}

\item Deployment strategies: Allows for user to define env variables and input arguments.
Can use file transfer to move input files around. Env variables are sourced
on the given node (resource?) where a task will run, and input arguments
are passed to the executable in much the same way as if it were typed
by the user on command line. Pilot-Manager (sec. 3 term: application?)
--- Pilot-Agent (sec. 3 term: pilot)
\end{itemize}

Auxiliary Capabilities:
\begin{itemize}
\item Architecture: Client/server; Agent (remote) / Manager (local). Centralized coordination server.

\item Interfaces: User interfaces with PilotJob System via the Pilot API, implemented in BigJob using the python programming language. Pilot, in most case, interfaces with target machine using the Simple API for Grid Applications (SAGA) [meaning PJ system does not have to translate PilotJob into scheduler-specific script], cloud adaptors are coded in python into BigJob.

\item Interoperability: Allows for interoperability with multiples types of resources (heterogenous schedulers and clouds/grids/clusters) at one time. Does not have interoperability across PilotJob systems

\item Multitenancy: No multitenancy. Single-user. 

\item Robustness: Depends on user for fault-tolerance, but can report job failures

\item Security: Follows resource security for scheduler systems (i.e. policies like allocations, etc). SSH auth. For clouds, allows for private/public key specific to VM

\item Usage Modes: Grids, clouds, clusters.
\end{itemize}

\paragraph{Rationale}
BigJob was designed as a flexible and extensible \pilotjob
to work on a variety of infrastructures. It was designed to
natively support many scientific applications, including
MPI jobs~\ref{saga_bigjob_condor_cloud}.

\paragraph{Design Goals}
BigJob supports the basic functionality required of a \pilotjob,
with the added capability of interfacing with a number of
different batch queuing systems and infrastructures (grids,
clouds, clusters, etc.). BigJob was also designed to
support MPI jobs without adding additional configuration
requirements to the end-user. Lastly, the application-level
programmability that BigJob offers was incorporated as a
means of giving the end-user more flexibility and control
over their job management.

Most recently, BigJob has been extended to work with data and, similarly to
\pilotjobs, abstract away direct user communication between different
storage systems\mrnote{Ref: Pilot Data}. This work has extended BigJob from being a purely
\pilotjob-based system to a more complete job and data management
system.

\paragraph{Applications}
BigJob has been used for many different ensemble-based or
replica-exchange-based applications in both the computational chemistry
and bioinformatics disciplines. It has been shown to work across multiple
XSEDE~\ref{xsede_url} machines and scale up to thousands of concurrent
jobs. It has also been used as the underlying job
management layer in the release of an asynchronous replica-exchange
software package~\ref{2013-xsede-cdi}.

\paragraph{Deployment Scenarios (Single User)}
BigJob is installable by a user onto the resource of his or her choice. It is capable of
running only in ``single user'' mode; that is, a \pilot belongs to the user who
spawned it and cannot be accessed by other users.

\paragraph{Resource Landscape (e.g. grid/cloud/hpc/etc)}
BigJob uses the Simple API for Grid Applications (SAGA)~\ref{ogf-gfd-90, sagastuff}
in order to interface with different grid middleware. It can work on HPC grid
environments, such as XSEDE or FutureGrid \mrnote{ref}, as well as personal
clusters with batch queuing systems. In addition, BigJob has been shown to work
on clouds~\ref{cpe_pmr_cloud_2012}, where it has the capability to both
launch VMs and submit jobs to these VMs.

\paragraph{Architecture \& Interface}

In the BigJob framework, a \pilot is, appropriately enough, called a
\pilot. The \pilot-\textit{Manager} is called a \textit{BigJob-
Manager}, which is the central coordinator of the
framework. The \textit{BigJob-Manager} is responsible for
the orchestration and scheduling of \pilots. It runs locally on the
machine used to initiate the distributed application which may or may
not be the same resource as the machine used on which the distributed
application executes.

The Manager ensures that tasks are launched onto the correct resource
using the correct number of processes. The BigJob-Manager submits a
\pilot to a remote resource's batch queueing system; when these
\pilots becomes active, a \textit{BigJob-Agent} on the remote resource 
gathers information about its resource
and executes the actual tasks the resource. The \textit{BigJob-Agent}
represents the \pilotjob, and thus, the application-level resource
manager on the respective resource.
 
The communication between the Manager and Agent(s) is achieved through
the {\it Distributed Coordination Service}, which is most often a
database that stores information about jobs, information about the
jobs (executable, input data, etc.), and job status. In the BigJob framework,
this Distributed Coordination Service is achieved through the use
of the \textit{redis} database\mrnote{cite redis}. 
The \pilot creation and management capabilities of BigJob meet 
the baseline capabilities for a \pilotjob as described in Section 3.


\paragraph{BigJob Conclusion}
Using SAGA has enabled BigJob to expand to many of the changing and evolving
architectures and middlewares. As a result, BigJob can grow to accomodate
new usage modes on heterogenous architectures. The uptake 
of BigJob by the scientific community
on both grid and cloud architectures shows the dynamism \pilotjobs offer.
The use of a \pilotjob system, such as BigJob, to marshal VMs as ``Pilots'' 
also lends itself to the extensibility of the \pilotjob concept.
\mrnote{It's late and I don't know if the previous sentence makes sense.
Read in morning}

\subsubsection{DIANE}



\paragraph{Rationale}

DIANE~\cite{Moscicki:908910} is a task coordination framework, which follows
the Master/Worker pattern.
It was developed at CERN for data analysis in the LHC experiment.
As DIANE is a framework \msnote{in contrast to a system? tool?}, some of its
semantics are not fixed, as they can be implemented in various ways.

The \textbf{Workload} 

The central component of DIANE is the \textit{RunMaster}.
It consists of a \textit{TaskScheduler} and an \textit{ApplicationManager}.
Both are abstract classes that need to be implemented for the specific purpose
at hand.
The \textit{TaskScheduler} keeps track of the task entries, maps the tasks to
the \textit{ApplicationWorkers}.
The implementation of the \textit{ApplicationManager} is as its name suggest,
responsible for defining the application \textbf{Workload} and the creating
\textbf{Tasks} that are passed to the \textit{TaskScheduler}.

The \pilot in DIANE is the \textit{WorkerAgent}.
The core component of the \textit{WorkerAgent} is the
\texttt{ApplicationWorker}, which is an abstract class that defines three
methods that every implementation needs to implement.
Two of these methods are for initialization and cleanup and the last
\texttt{do\_work()} is the method that actually receives the task description
and executes the work.

The \textbf{Robustness} in DIANE comes from the mature CORBA communication
layer, and from custom task-failure policies in the \textit{TaskScheduler}.

As expected from the fact that DIANE is in essence a Master/Worker framework,
the \textbf{Tasks} have inside the framework no relationships, thereby
resembling a Bag-of-Tasks. 

The \textbf{Coordination \& Communication} in DIANE is based on CORBA using
TCP/IP.
CORBA itself is invisible to the application.
Networking-wise, the workers are clients of the master server.
On low-level, communication is always uni-directionally from the
\textit{WorkerAgent} to the \textit{RunMaster}.
This implies that the network requirements are such that the
\textit{WorkerAgent} needs to be able to reach the \textit{RunMaster} through
TCP/IP.
Bi-directional communication is achieved by periodic polling through
heartbeats by the \textit{WorkerAgent}, where the \textit{RunMaster} responds
with feedback.
The authors report that in first instance they had implemented full
bi-directional communication, but that turned out to be difficult to correctly
implement and created scalability limitations.




\paragraph{Design Goals}




DIANE utilizes a single hierarchy of worker agents as well as a PJ \textit{manager}
referred to as \texttt{RunMaster}.  The manager creates
placeholders called \texttt{Workers}, which register with and receive jobs from the manager.
%For the spawning of PJs a separate script, the so-called submitter script, is
%required.
The \textit{scheduler} for DIANE is known as the \texttt{Planner}.
%For the access to the physical resources the GANGA
%framework~\cite{Moscicki20092303} can be used.
%A worker agent generally manages only a single
%core and thus, by default is not able to run parallel applications.
% Once the worker agents are started they register themselves at the RunMaster.
% GANGA provides a
% unified interface for job submissions to various resource types, e.\,g.\ EGI
% resources or TG resources via a SAGA backend.
%DIANE operates on the master-worker computing model.

%The \textit{manager} for DIANE is called the \texttt{Master}, and contains
%the \textit{scheduler} for DIANE, which is known as the \texttt{Planner}.
%The manager creates placeholders called \texttt{Workers}, which
%register with and receive jobs from the manager.

Further, DIANE supports fault tolerance: basic error detection and
propagation mechanisms are in place. Further, an automatic re-execution of WUs
is possible.

\paragraph{Applications}

DIANE was originally conceived for HEP applications at CERN, but has since been
used for various other domains, though mainly in Life Sciences.

\paragraph{Deployment Scenarios (Single User)}


\msnote{Whats our definition of single/multi user? Is it that multiple users
can make use of one pilot, or that the pilot runs as a different user on the
resource?}


\paragraph{Deployment Scenarios (Multi User/VO)}



\paragraph{Architecture \& Interface}

DIANE's architecture is based on the \textit{Inversion of Control} design pattern.
In DIANE's case, this means that it is a python-based framework, that
formulates certain hooks that an "application" can be programmed against.
These DIANE-applications are then started through a diane-run command.

The workflow is then such that a user submits a parallel job to
the grid using a DIANE client.  The client then creates the Pilot manager
and remains in contact with it to control its \pilotjobs.  Results
are aggregated through use means of the DIANE
\texttt{Integrator}.\aznote{No mapping vocab for the integrator as of now.}
DIANE includes a simple capability matcher and FIFO-based task scheduler
to help facilitate the execution of jobs.
Plugins for other workloads, e.\,g.\ DAGs or for data-intensive
application, exist or are under development. The framework is extensible:
applications can implement a custom application-level scheduler.

For communication between the RunMaster and
worker agents point-to-point messaging based on CORBA~\cite{OMG-CORBA303:2004}
is used. CORBA is also used for file staging.
DIANE is a single-user PJ, i.\,e.\ each PJ is executed with the
privileges of the respective user. Also, only WUs of this respective user can be
executed by DIANE. DIANE supports various middleware security mechanisms
(e.\,g.\ GSI, X509 authentication). For this purpose it relies on GANGA. The
implementation of GSI on TCP-level is possible, but currently not yet
implemented. 


\paragraph{Resource Landscape (e.g. grid/cloud/hpc/etc)}

DIANE is primarily designed with respect to HTC environments (such as
EGI~\cite{egi}), i.\,e.\ one PJ consists of a single worker agent with the
size of 1 core.

\paragraph{DIANE Conclusion}



%\onote{OLE'S SECTION}

%\onote{It seems that BOSCO is some sort of a user-space version of Condor.
%It can interface with single clusters as well as complex HTC grids (GlideinWMS).
%Multiple resource-scanrios are possible. There is a bosco-submit node which
%holds the user jobs. Bosco uses Condor glidein (-agents) internally as a
%resource overlay. The glideins pull data from the bosco-submit node. The
%main difference between BOSCO and Condor (even though both expose the same
%user API) is that BOSCO allows ad-hoc usage, while Condor requires a rather
%complex setup. In that regard, BOSCO is somehwat similar to BigJob.
%This page is somehwat insightful: http://bosco.opensciencegrid.org/about/}

%% \aznote{Section copied and pasted from IPDPS paper for now -- plan
%% to use this general sort of information + reword using our spiffy
%% vocabulary presented in this paper}
%% % Coordination and Communication
%% DIANE~\cite{Moscicki:908910} is a task coordination framework, which
%% was originally designed for implementing master/worker applications,
%% but also provides PJ functionality for job-style executions. DIANE
%% utilizes a single hierarchy of worker agents as well as a PJ manager
%% referred to as \texttt{RunMaster}.
%% %Further, there is ongoing work on a multi-master extension.
%% For the spawning of PJs a separate script, the so-called submitter script, is
%% required. For the access to the physical resources the GANGA
%% framework~\cite{Moscicki20092303} can be used.
%% %GANGA provides a
%% %unified interface for job submissions to various resource types, e.\,g.\ EGI
%% %resources or TG resources via a SAGA backend.
%% Once the worker agents are started they register themselves at the RunMaster.
%% In contrast to TROY-BigJob, a worker agent generally manages only a single
%% core and thus, by default is not able to run parallel applications (e.\,g.\
%% based on MPI). BJ utilizes the BJ-Agent that is able manage a set of local
%% resources (e.\,g.\ a certain number of nodes and cores) and thus, is capable
%% of running parallel applications. For communication between the RunMaster and
%% worker agents point-to-point messaging based on CORBA~\cite{OMG-CORBA303:2004}
%% is used. CORBA is also used for file staging, which is not fully supported by
%% BJ, yet.

%% % Binding
%% DIANE is primarily designed with respect to HTC environments (such as
%% EGI~\cite{egi}), i.\,e.\ one PJ consists of a single worker agent with the
%% size of 1 core. BJ in contrast is designed for HPC systems such as TG,
%% where a job usually allocates multiple nodes and cores. To address this issue
%% a so-called multinode submitter script can be used: the scripts starts a
%% defined number of worker agents on a certain resource. However, WUs will be
%% constrained to the specific number of cores managed by a worker agent. A
%% flexible allocation of resource chunks as with BJ is not possible. By
%% default a WU is mapped to a SU; application can however implement smarter
%% allocation schemes, e.\,g.\ the clustering of multiple WUs into a SU.

%% %Scheduling
%% DIANE includes a simple capability matcher and FIFO-based task scheduler.
%% Plugins for other workloads, e.\,g.\ DAGs or for data-intensive
%% application, exist or are under development. The framework is extensible:
%% applications can implement a custom application-level scheduler.


%% %Other impl. related issues: FT and security
%% DIANE is, just like BJ, a single-user PJ, i.\,e.\ each PJ is executed with the
%% privileges of the respective user. Also, only WUs of this respective user can be
%% executed by DIANE. DIANE supports various middleware security mechanisms
%% (e.\,g.\ GSI, X509 authentication). For this purpose it relies on GANGA. The
%% implementation of GSI on TCP-level is possible, but currently not yet
%% implemented. Further, DIANE supports fault tolerance: basic error detection and
%% propagation mechanisms are in place. Further, an automatic re-execution of WUs
%% is possible.


%% \aznote{Section copied and pasted from IPDPS paper for now -- plan
%% to use this general sort of information + reword using our spiffy
%% vocabulary presented in this paper}
%% Swift~\cite{Wilde2011} is a scripting language designed for expressing
%% abstract workflows and computations. The language provides, amongst many other
%% things, capabilities for executing external applications, as well as the
%% implicit management of data flows between application tasks.
%% % For this
%% % purpose, Swift formalizes the way that applications can define
%% % data-dependencies. Using so called mappers, these dependencies can be
%% % easily extended to files or groups of files.
%% The runtime environment handles the allocation of resources and the spawning of
%% the compute tasks.
%% % Both data- and execution management capabilities are provided
%% % via abstract interfaces.
%% Swift supports e.\,g.\ Globus, Condor and PBS resources.
%% % The pool of resources
%% % that is used for an application is statically defined in a configuration file.
%% % While this configuration file can refer to highly dynamic resources (such as OSG
%% % resources), there is no possibility to manage this resource pool
%% % programmatically.
%% By default, Swift uses a 1:1 mapping for \cus and \sus. However,
%% Swift supports the grouping of SUs as well as PJs. For the PJ functionality, Swift uses the
%% Coaster~\cite{coasters} framework. Coaster relies on a master/worker
%% coordination model; communication is implemented using GSI-secured TCP sockets.
%% Swift and Coaster support various scheduling mechanisms, e.\,g.\ a FIFO and a
%% load-aware scheduler. Additionally, Swift can be used in conjunction with
%% Falkon~\cite{1362680}, which also provides \pilot-like functionality.

%% % Falkon
%% % refers to \pilots as the so called provisioner, which are created using the
%% % Globus GRAM service. The provisioner spawns a set of executor processes on the
%% % allocated resources, which are then responsible for managing the execution of
%% % SUs. \cus are submitted via a so called dispatcher service. Similar to Coaster,
%% % Falkon utilizes a M/W coordination model, i.\,e.\ the executors periodically
%% % query the dispatcher for new SUs. Web services are used for communication.

%\subsubsection{GWPilot}
%% \aznote{Considering the new direction of the paper, I am not sure whether
%% GWPilot~\cite{gwpilot} requires its own section for analysis...}

%% %\begin{lstlisting}[breaklines]
%% %\url{https://indico.egi.eu/indico/materialDisplay.py?contribId=18&sessionId=46&materialId=slides&confId=1019}
%% %\url{http://ieeexplore.ieee.org/xpls/abs_all.jsp?arnumber=6266981}
%% \begin{itemize}
%% \item Integrates with GridWay metascheduler
%% \item Pilots advertise to GridWay, GridWay scheduler schedules pilots
%% \item Pilots pull tasks from scheduler
%% \item Installation as a GridWay driver -- written in Python
%% \item Interoperability managed by GridWay drivers (DRMAA, JDSL, BES, more?)
%% \item Using GWPilot requires only adding a single line to their GridWay task
%% \item ``Lightweight and scalable''
%% \end{itemize}
%% %\end{lstlisting}

%\subsubsection{Bosco}
%% \aznote{Given the modified direction we are taking the paper (``landmark''
%% pilot-jobs reviewed as opposed to all-inclusive), I suggest nixing
%% the inclusion of Bosco~\cite{bosco} in this section as Bosco appears
%% to be mostly making strides with regards to user-accessibility
%% of pilots.}
%% \\
%% \begin{itemize}
%% \item Condor used as batch system/user interface
%% \item Single submit model for different cluster types (LSF/PBS/SGE/etc) via SSH with \texttt{BLAHPD}
%% \item Campus Factory (condor overlay generator) creates glideins, checks users queue for idle jobs,
%%   enforces submission policies
%% \item Bosco = ``BLAHPD Over SSH Condor Overlay''
%% \item Workstation-based (run Bosco client locally)
%% \item Multi-user (Bosco workstation install can be used by multiple researchers)
%% \item Supports multiple cluster submission (but what about coordination...)

%% \end{itemize}
% \upp
% \subsection{SWIFT-Coaster\upp\upp}
%
% SWIFT~\cite{Wilde2011} is a scripting language designed for expressing abstract
% rest of this cut, but making a note of this in case we want
% to bring swift into the discussion later that i can find more info in 2011 paper
%\aznote{Much of what I was thinking of doing is similar to the 2011 IPDPS paper
%in the 2012-PStar directory.  Is this along the right lines?}

%\aznote{A bit confused about how to tie this into conclusion and discussion:
%``The need for a common minimum model'' in particular.  I am guessing that
%I should help to ``set up'' that section by providing plenty of material showing
%that the pilot job systems are very similar at their core.  How far should I
%go with this, however, to avoid hitting the ``conclusion'' material in this section?
%}

\section{Conclusion and Discussion}\label{sec:5}


\jhanote{the structure of section 5 is: (i) revisit the myths. (ii)
  state clearly how section 3 and 4 help us define what a pilot is,
  the necessary and sufficient conditions (if possible), discuss the
  classifiers and apply/discuss them to the set of pilot systems here.
  (iii) then we go on to say motivate P*/pilot for data, (iv) discuss
  implications for WF systems and conclude with a summary for
  tools/sustainability/etc. At some point, we discuss how/why pilots
  are more than just pilots, eg can be RM layer for middleware,
  runtime framework etc}



\jhanote{Main message is: (i) \pilotjobs have potential, but it is not
  being realized due to ad hoc nature of theory and practise, (ii) we
  provide first comprehensive historical and technical analysis, (iii)
  set the stage for a common conceptual model and implementation
  framework, and (iv) provide insight and lessons for other tools and
  higher-level frameworks, such as Workflow systems possibly}

\mtnote{Lifted from Section 3}

It should be noted that cloud-based DCIs introduce notable exceptions and
differences in the way in which pilots can be provisioned.[cit, cit]
Within a IaaS [cit], Virtual Machines (VMs) and not jobs are used for
their provisioning. VMs can often be instantiated without waiting into
a queue, and limitations on the execution time of a VM can be
virtually absent. Clearly, overheads are introduced by having to deal
with VMs and not simple jobs and the model adopted within a IaaS-based
DCI to assign resources to each VM can affect the flexibility of the
whole \pilotjob system. [cit] A similar assessment could be done for a DCI
deploying a PaaS model of cloud computing.[cit]

\mrnote{Though you avoid queue wait, you introduce VM start-up time overhead
as well}

Another important detail to notice is that while the operations of a
task performed by means of a \pilot are usually computational in
nature, in principle, they might also operate on other type of
resources such as data or network bandwidth, depending on the
resources made available by an infrastructure to its users. In this
context, it would be likely for some \pilotjob systems to use the
terms `pilot data' or `pilot network'.

\jhanote{this should move to much later in this section or maybe to Section 5.
Must bring clarity to compute before opening other cans of worms}


\subsection{revisit the ``myths''}

\aznote{From the e-mail chain:
So the top 5 ``myths'' are:
1) PJs are just for HTC and they just circumvent job queuing delays;
PJs unfairly game HPC queuing system
2) PJs are such a simple concept, it doesn't need more attention:
Conversely, everyone should write their own PJ just because they can
3) PJs have to be tied to specific infrastructure and infrastructures have
to be tied to specific pjs
4) PJs are stand-alone tools passive (system) tools, as opposed to
user-space, active and extensible components of a CI
5) PJs have well defined semantics and model. PJs dont help with data
intensive applications
}

\subsection{The need for a common minimum model}

``pilot-abstractions works!'' , p* as a model ok.

%Whereas we will discuss in greater detail some of the concepts upon
%which this paper is built, for completeness we briefly outline them
%here.

Our initial investigation~\cite{Luckow:2008la} into
\pilot-Abstractions was motivated by the desire to provide a single
conceptual framework --- referred to as the P* Model, that would be
used to understand and reason the plethora and myriad \pilotjob
implementations that exist.


Once a common and uniform conceptual model was available, the notion
of \pilotdata was conceived using the power of symmetry, i.e., the
notion of \pilotdata was as fundamental to dynamic data placement and
scheduling as \pilotjobs was to computational tasks. As a measure of
validity, the \pstar model was amenable and easily extensible to
\pilotdata.  The consistent and symmetrical treatment of data and
compute in the model led to the generalization of the model as the
{\it P* Model of Pilot Abstractions}.


\subsection{Lessons for Workflow System}

The current state of workflow (WF) systems~\cite{nsf-workflow,1196459}
provides a motivating example for the P* Model and the Pilot-API: even
though many WF systems exist (with significant duplicated effort),
they provide limited means for extensibility and interoperability.  We
are not naive enough to suggest a single reason, but assert that one
important contributing fact is the lack of the right interface
abstractions upon which to construct workflow systems; had those been
available, many/most WF engines would have likely utilized them (or
parts thereof), instead of proprietary solutions.

% That would not immediately allow WF implementations to interoperate,
% but would make semantic mapping between them significantly simpler,
% thus supporting the very notion of interoperation.

Significant effort has been invested towards WF interoperability at
different levels -- if nothing else, providing post-facto
justification of its importance. The impact of missing interface
abstractions on the WF world can be seen through the consequences of
their absence: WF interoperability remains difficult if not
infeasible. The Pilot-API in conjunction with the P* Model aims to
prevent similar situation for \pilotjobs.

\note{What is the future of PJ? Why should we do to enhance the usability?}
\aznote{I would argue that our work is important to \pilotjobs because
by expressing a common model, we enable researchers to 1) understand the
commonalities between existing \pilotjob approaches in order to 2) motivate
innovation + construction of ``next-gen'' \pilotjob systems.  E.G.,
implement the basics (or work from an existing system) + understand
where the boundaries/unexplored territory is without having to first
completely understand all 15+ existing \pilotjob systems and their
unique vocabulary.}
%\footnote{To be fair: we are not sure if a generic model and/or a
%  generic WF API are achievable {\it on a useful level} -- we think,
%  nevertherless, that our discussion is valid.}



%\mrnote{We have a survey of related \pilotjobs systems, a 'history'
%of \pilotjobs, and a related work section? Not sure if we should
%condense in some way. PS Why is this after conclusion?}

%\subsection{Scientific Data Management}

\section*{The section for Unresolved Ideas and Issues}

\jhanote{The question is what are the fundamental ``concepts''. It is
  not necessary that the concepts have a specific implementation or
  map to a component in \pilotjob system. As we know there are
  different ways in which tasks get committed to the \pilotjob
  system. One possible primary concept is that of logical grouping;
  all tasks are committed to a logical grouping -- where the grouping
  is such that all entities in this group will be executed
  (disregrarding details of how this grouping will happen, or who will
  perform the execution).  It appears that the concept of logical
  grouping of tasks is a fundamental one, and avoids a requirement of
  any further specification of of details of who/where/when; if so,
  then the notion of a pool can be dispensed with, which will have the
  advantage of liberating us from the requirement of imposing on the
  the manager the need to push/pulling tasks from a pool.}\mtnote{The
  paragraph relative to this comment is gone --- no more pool
  concept. I do like the idea of grouping though so I would suggest to
  wait for the entire Section to stabilize further and then see
  whether we can add/extend a paragraph by introducing the concept of
  grouping.}

\jhanote{AppLeS is not strictly \pilotjob based?  but \pilotjob like
  capabilities?} \alnote{The question is: when is a \pilot a \pilot?
  When the use the term \pilot and when \pilot-like? Apples has a
  component -- the Actuator (Quote from paper: "... handles task
  launching, polling, and cancellation ..."), which is quite similar
  to a \pilot. But maybe AppLeS is something for the history section
  or a separate category for Master/Worker frameworks...  PJ evolved
  from the need to map master/worker style computations to
  heterogeneous, dynamic distributed grid environments. Added a
  pre-\pilot category to the history sub-section.}

\section{Discussion Area}

\jhanote{I think all the functionality of PJs is predicated on the
  following core capability: enable the decoupling between assigning a
  workload from the spatial/temporal execution properties of its
  execution. In condor, this is mentioned as ``separating between job
  scheduling and resource''. This get mentioned somewhere in core
  functionality?}

\jhanote{deployment and provisioning are not the same in my
  opinion: provisioning is about resources, i.e., provisioning is not
  the same as scheduling, it is more like arranging. Deployment is
  about ``setting up T=0 requirements, which could be software
  environment, input/data dependencies, as well as resources.''}

\aznote{Some additional thoughts for this section...
1) The early history of \pilotjobs seems to have a few ``independent''
\pilotjob approaches, where systems (eg MyCluster/AppLeS) incorporate
\pilotjob techniques with sometimes wildly differing 
implementations/vocabulary but when boiled down, offering the same
attempts at base/minimal functionality.  This helps push the
value of the approach (multiple groups working on it independently
helps imply its value), with increasingly complex \pilotjob systems
coming out as time pushes on.  Our review aids both in the
construction of these increasingly complex systems by exposing
their core functionality + allowing researchers/etc to push
the field forward by reducing the apparent complexity to a set
of common terms/comparisons/classifiers/etc.  I don't know if this is
too prescriptive/hand-wavey -- feel free to ignore if so, but I feel that 
the essence of this is one
potential contribution of our work.}

% jhanote{I have brought the above sentence down here. I think for ease
%   of readability, once we start discussing binding we should continue
%   onto early and late binding. Also in the definition of early and
%   late binding, we should IMHO explicitly use the concept of ``when''
%   a task is bound to a resource in order for it to be concrete.}

% \mtnote{yet another way to define early/late binding. I would suggest
%   to discuss about the definitions of these two concepts at our next
%   meeting before attempting to write any conclusive definition.}

% The act of assigning resources to tasks is called `binding'. Note how
% `binding' is defined in contrast to scheduling.

% \jhanote{How about?} Assignment is the notion of commiting a task (or
% a pilot) to a specific resource. Binding is the process of
% constraining a task (or a pilot) at a specific point in time to a
% resource; Furthermore, scheduling is the process of commiting a task
% that has been bound to a resource to a specific execution queue. The
% order of these atomic operations may differ.

%\onote{This section tries to define everything from concepts (e.g., multi-level-scheduling) to specific architectural components (agents, manager). It's hard to parse and I doubt that it will be particularly useful to the reader. I think it would make sense to have a stricter distinction (e.g., by using sub-sections) between concepts (late-binding, early-binding, multi-level) and architectural building blocks that are common to all PJ-Systems. On that note: why are we trying to come up with this arbitrary vocabulary (on the `building block' side), consisting of `agents' and `managers' at all? After all, PJs are `just' an instance of Master-Worker with specific properties, capabilities and operating in a specific environment. I think shooting from that angle (at least making an attempt to) might help to introduce more clarity. I think this would be more intuitive from a reader's perspective. And after all, it's about unlearning P*, isn't it?}  \mtnote{Thank you Ole! Here a completely rewrite of the Section that tries to address the points you made.}


\section*{Acknowledgements}
{\footnotesize{This work is funded by NSF CHE-1125332 (Cyber-enabled
  Discovery and Innovation), NSF-ExTENCI (OCI-1007115) and
  ``Collaborative Research: Standards-Based Cyberinfrastructure for
  Hydrometeorologic Modeling: US-European Research Partnership''
  (OCI-1235085) and Department of Energy Award (ASCR)
  DE-FG02-12ER26115.  This work has also been made possible thanks to
  computer resources provided by TeraGrid TRAC award TG-MCB090174 and
  BiG Grid.  This document was developed with support from the US NSF
  under Grant No. 0910812 to Indiana University for ``FutureGrid: An
  Experimental, High-Performance Grid Test-bed''.}}

% \bibliographystyle{IEEEtran}
\bibliographystyle{abbrv}
\bibliography{pilotjob,literatur,saga,saga-related,urls}




\end{document}

